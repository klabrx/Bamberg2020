\section{Referenzen}
\subsection{Rechtliches}
\begin{frame}{Gesetzestexte}
    \begin{itemize}
        \item \href{https://ec.europa.eu/eurostat/de/web/products-statistical-books/-/KS-31-09-254}{EU-Statistikverordnung}
        \item \href{https://www.gesetze-im-internet.de/bstatg_1987/BStatG.pdf}{Bundesstatistikgesetz}
        \item \href{http://www.landesrecht-bw.de/jportal/portal/t/sx7/page/bsbawueprod.psml/screen/JWPDFScreen/filename/StatG_BW_jlr-StatGBWrahmen.pdf}{LStatG - (BaWü)}
        \item \href{https://www.gesetze-bayern.de/Content/Pdf/BayStatG?all=True}{BayStatG}
        \item \href{https://www.gesetze-im-internet.de/bdsg_2018/BDSG.pdf}{Bundesdatenschutzgesetz}
    \end{itemize}
\end{frame}

\begin{frame}[allowframebreaks]{Art. 46 BDSG - Begriffsbestimmungen}
\begin{enumerate}
    \item \alert{personenbezogene Daten} alle Informationen, die sich auf eine identifizierte oder identifizierbare natürliche Person (betroffene Person) beziehen; als identifizierbar wird eine natürliche Person angesehen, die direkt oder indirekt, insbesondere mittels Zuordnung zu einer Kennung wie einem Namen, zu einer Kennnummer, zu Standortdaten, zu einer Online-Kennung oder zu einem oder mehreren besonderen Merkmalen, die Ausdruck der physischen, physiologischen, genetischen, psychischen, wirtschaftlichen, kulturellen oder sozialen Identität dieser Person sind, identifiziert werden kann;
    \item \alert{Verarbeitung} jeden mit oder ohne Hilfe automatisierter Verfahren ausgeführten Vorgang oder jede solche Vorgangsreihe im Zusammenhang mit personenbezogenen Daten wie das Erheben, das Erfassen, die Organisation, das Ordnen, die Speicherung, die Anpassung, die Veränderung, das Auslesen, das Abfragen, die Verwendung, die Offenlegung durch Übermittlung, Verbreitung oder eine andere Form der Bereitstellung, den Abgleich, die Verknüpfung, die Einschränkung, das Löschen oder die Vernichtung;
    \item \alert{Einschränkung der Verarbeitung} die Markierung gespeicherter personenbezogener Daten mit dem Ziel, ihre künftige Verarbeitung einzuschränken;
    \item \alert{Profiling} jede Art der automatisierten Verarbeitung personenbezogener Daten, bei der diese Daten verwendet werden, um bestimmte persönliche Aspekte, die sich auf eine natürliche Person beziehen, zu bewerten, insbesondere um Aspekte der Arbeitsleistung, der wirtschaftlichen Lage, der Gesundheit, der persönlichen Vorlieben, der Interessen, der Zuverlässigkeit, des Verhaltens, der Aufenthaltsorte oder der Ortswechsel dieser natürlichen Person zu analysieren oder vorherzusagen;
    \item \alert{Pseudonymisierung} die Verarbeitung personenbezogener Daten in einer Weise, in der die Daten ohne Hinzuziehung zusätzlicher Informationen nicht mehr einer spezifischen betroffenen Person zugeordnet werden können, sofern diese zusätzlichen Informationen gesondert aufbewahrt werden und technischen und organisatorischen Maßnahmen unterliegen, die gewährleisten, dass die Daten keiner betroffenen Person zugewiesen werden können;
    \item \alert{Dateisystem} jede strukturierte Sammlung personenbezogener Daten, die nach bestimmten Kriterien zugänglich sind, unabhängig davon, ob diese Sammlung zentral, dezentral oder nach funktionalen oder geografischen Gesichtspunkten geordnet geführt wird;
    \item \alert{Verantwortlicher} die natürliche oder juristische Person, Behörde, Einrichtung oder andere Stelle, die allein oder gemeinsam mit anderen über die Zwecke und Mittel der Verarbeitung von personenbezogenen Daten entscheidet;
    \item \alert{Auftragsverarbeiter} eine natürliche oder juristische Person, Behörde, Einrichtung oder andere Stelle, die personenbezogene Daten im Auftrag des Verantwortlichen verarbeitet;
    \item \alert{Empfänger} eine natürliche oder juristische Person, Behörde, Einrichtung oder andere Stelle, der personenbezogene Daten offengelegt werden, unabhängig davon, ob es sich bei ihr um einen Dritten handelt oder nicht; Behörden, die im Rahmen eines bestimmten Untersuchungsauftrags nach dem Unionsrecht oder anderen Rechtsvorschriften personenbezogene Daten erhalten, gelten jedoch nicht als Empfänger; die Verarbeitung dieser Daten durch die genannten Behörden erfolgt im Einklang mit den geltenden Datenschutzvorschriften gemäß den Zwecken der Verarbeitung;
    \item \alert{Verletzung des Schutzes personenbezogener Daten} eine Verletzung der Sicherheit, die zur unbeabsichtigten oder unrechtmäßigen Vernichtung, zum Verlust, zur Veränderung oder zur unbefugten Offenlegung von oder zum unbefugten Zugang zu personenbezogenen Daten geführt hat, die verarbeitet wurden;
    \item \alert{genetische Daten} personenbezogene Daten zu den ererbten oder erworbenen genetischen Eigenschaften einer natürlichen Person, die eindeutige Informationen über die Physiologie oder die Gesundheit dieser Person liefern, insbesondere solche, die aus der Analyse einer biologischen Probe der Person gewonnen wurden;
    \item \alert{biometrische Daten} mit speziellen technischen Verfahren gewonnene personenbezogene Daten zu den physischen, physiologischen oder verhaltenstypischen Merkmalen einer natürlichen Person, die die eindeutige Identifizierung dieser natürlichen Person ermöglichen oder bestätigen, insbesondere Gesichtsbilder oder daktyloskopische Daten;
    \item \alert{Gesundheitsdaten} personenbezogene Daten, die sich auf die körperliche oder geistige Gesundheit einer natürlichen Person, einschließlich der Erbringung von Gesundheitsdienstleistungen, beziehen und aus denen Informationen über deren Gesundheitszustand hervorgehen;
    \item \alert{besondere Kategorien personenbezogener Daten}
        \begin{itemize}
            \item Daten, aus denen die rassische oder ethnische Herkunft, politische Meinungen, religiöse oder weltanschauliche Überzeugungen oder die Gewerkschaftszugehörigkeit hervorgehen,
            \item genetische Daten,
            \item biometrische Daten zur eindeutigen Identifizierung einer natürlichen Person,
            \item Gesundheitsdaten und
            \item Daten zum Sexualleben oder zur sexuellen Orientierung;
        \end{itemize}
    \item \alert{Aufsichtsbehörde} eine von einem Mitgliedstaat gemäß Artikel 41 der Richtlinie (EU) 2016/680 eingerichtete unabhängige staatliche Stelle;
    \item \alert{internationale Organisation} eine völkerrechtliche Organisation und ihre nachgeordneten Stellen sowie jede sonstige Einrichtung, die durch eine von zwei oder mehr Staaten geschlossene Übereinkunft oder auf der Grundlage einer solchen Übereinkunft geschaffen wurde;
    \item \alert{Einwilligung} jede freiwillig für den bestimmten Fall, in informierter Weise und unmissverständlich abgegebene Willensbekundung in Form einer Erklärung oder einer sonstigen eindeutigen bestätigenden Handlung, mit der die betroffene Person zu verstehen gibt, dass sie mit der Verarbeitung der sie betreffenden personenbezogenen Daten einverstanden ist.
\end{enumerate}

\end{frame}