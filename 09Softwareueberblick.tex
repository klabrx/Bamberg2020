\section{Softwareoptionen}
\begin{frame}{Standardsoftware}
    \begin{itemize}
        \item Office-Pakete sind allgegenwärtig, für den Produktiveinsatz grundsätzlich auch geeignet. Limits bei der Anzahl der Zeilen gibt es eigentlich nicht mehr. Gestaltungsmöglichkeiten für Berichte, Tabellen und Diagramme sind durchaus brauchbar.
        \item An DB-Software (Oracle, MS-SQL, mySQL, \dots) führt eigentlich kein Weg vorbei.
    \end{itemize}

\end{frame}

\begin{frame}{Spezielle Statistiksoftware}
\begin{itemize}
    \item SPSS (bzw. der OpenSource-Clon PSPP) sind in vielen Städten im Einsatz, jedoch in nicht unerheblichem Ausmaß kostenpflichtig
    \item R scheint sich in den letzten Jahren an den Hochschulen immer mehr durchzusetzen
    \item Seit 2019 gibt es im VDSt die Kosis-Gemeinschaft ``Ko-R'', die zum Ziel hat, R als gemeinsame Entwicklungs- und Anwendungssprache zu etablieren.
\end{itemize}

\end{frame}