\section{Datenlogistik}

  \begin{frame}{Daten -> Informationen -> Wissen}
  \metroset{block=fill}
    \begin{block}{Daten}
      Daten sind zunächst kontextlose Folgen von Zeichen, Ziffern, Hieroglyphen, Symbolen, Emojies ...
    \end{block}
    \begin{block}Informationen}
      Werden Daten in einen Kontext\footnote{Dieser Kontext kann aus einer Satzbeschreibung bestehen.} gestellt oder lösen sie weitere Konsequenzen aus, werden sie zu Informationen: Informationen sind Daten mit Kontext.
    \end{block}
    \begin{block}{Wissen}
      Werden Informationen mit anderen Informationen verknüpft oder mit Angaben zu Quellen, Zuverlässigkeit, Entstehungsmethodik, \dots verbunden, entsteht aus Informationen Wissen: Wissen ist geprüfte Information.
    \end{block}
  \end{frame}
  \begin{frame}{Das informelle Informationsgesetz}
    \begin{description}
      \item[Fakten =] Information - Emotion
      \item[Meinung =] Information + Erfahrung
      \item[Ignoranz =] Meinung - Information
      \item[Dummheit =] Meinung $\neq$ Fakten      
    \end{description}
  \end{frame}

  \begin{frame}{Einwohnerdaten}
  \metroset{block=fill}
    \begin{block}{Standard-Datensätze}
      \begin{description}
        \item[Quelle:] Exportroutine im Melderegister
        \item[Abruf:] Zwei Optionen
          \begin{itemize}
            \item ``Dauerauftrag'' im vorgegebenen Turnus: Datenlieferung entsteht im Meldeamt, Transportweg ist festzulegen und zu sichern
            \item Eigener Abruf bei Bedarf: Zugang der Statistik zum EWO ist erforderlich, Exportdaten entstehen dafür schon im abgesicherten Bereich.
          \end{itemize}
         \item[Plausibilisierung:] Überprüfung/Korrektur von Schreibweisen, Zahlendrehern, evtl. Löschung unerwünschter Datensätze
         \item[Anreicherung:] Erzeugung zusätzlicher Informationen\footnote{Spätestens ab diesem Zeitpunkt unterliegen die Daten dem Rückspielverbot.}  wie Haushaltszugehörigkeit, Migrationshintergrund
        \end{description}
    \end{block}
  \end{frame}
