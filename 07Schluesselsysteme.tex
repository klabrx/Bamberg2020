\section{Schlüsselsysteme}
    \begin{frame}{Beispiele}
        \begin{description}
            \item[WZ2008:] Klassifikation der Wirtschaftszweige
            \item[Nationalitätenschlüssel:] Staats- und Gebietssystematik
            \item[AGS:] Gemeindeverzeichnis-Informationssystem GV-ISys
        \end{description}
    \end{frame}
    
    \begin{frame}{WZ2008}
        \begin{itemize}
            \item Grundlage der Aggregation volkswirtschaflticher Gesamtrechnungen
            \item veröffentlicht von destatis (im Verbund mit einer Vielzahl internationaler Klassifikationssysteme \dots die aktuelle Veröffentlichung umfasst 828(!) Seiten)
            \item hierarchisch aufgebaut (z.B.) im Abschnitt P (Erziehung und Unterricht):
                \begin{description}
                    \item[85] Erziehung und Unterricht 
                    \item[85.5] Sonstiger Unterricht
                    \item[85.59] Sonstiger Unterricht a.n.g.\footnote{``anderweitig nicht genannt''}
                    \item[85.59.2] Berufliche Erwachsenenbildung
                \end{description}
        \end{itemize}
      
    \end{frame}
    \begin{frame}{Nationalitätenschlüssel}
        \begin{itemize}
            \item veröffentlicht von destatis (in internationaler Abstimmung)
            \item enthält:
            \begin{itemize}
                \item Staatsnamen in Kurzform (``Vatikanstadt''),
                \item \dots und in Vollform (``Staat Vatikanstadt''),
                \item Staatsangehörigkeit (``vatikanisch''),
                \item  Kontinent (``EUR''),
                \item zwei- und dreistelliger ISO-3166-1-Code (`VA''\footnote{Genutzt insb. als Top Level Domain (TLD) im WWW}, ``VAT''),
                \item Destatis-BEV-Code (``167'')\footnote{Dieser BEV-Code entspricht dem in der Bevölkerungsstatistik verwendeten Nationalitätenschlüssel im engeren Sinn, nicht zu verwechseln mit anderen Schlüsseln, die für den Internationalen Handel verwendet werden.} 
            \end{itemize}    
        \end{itemize}
    \end{frame}
    \begin{frame}{Aufbau des BEV-Code}
        \begin{description}
            \item[000:] Deutschland, somit
            \item[(1-5)xx] Ausland, geordnet nach Kontinenten:
                 \begin{description}
                    \item[1xx:] Europa
                    \item[2xx:] Afrika
                    \item[3xx:] Amerika(s)
                    \item[4xx:] Asien
                    \item[5xx:] Australien und Ozeanien
                 \end{description}
            
        \end{description}
        
    \end{frame}
    
    \begin{frame}{GV-ISys}

    \end{frame}
    