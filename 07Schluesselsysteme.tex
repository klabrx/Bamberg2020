\section{Schlüsselsysteme}
    \begin{frame}{Beispiele}
        \begin{description}
            \item[WZ2008:] Klassifikation der Wirtschaftszweige
            \item[Nationalitätenschlüssel:] Staats- und Gebietssystematik
            \item[AGS:] Gemeindeverzeichnis-Informationssystem GV-ISys
        \end{description}
    \end{frame}
    
    \begin{frame}{WZ2008}
        \begin{itemize}
            \item Grundlage der Aggregation volkswirtschaflticher Gesamtrechnungen
            \item veröffentlicht von destatis (im Verbund mit einer Vielzahl internationaler Klassifikationssysteme \dots die aktuelle Veröffentlichung umfasst 828(!) Seiten)
            \item hierarchisch aufgebaut (z.B.) im Abschnitt P (Erziehung und Unterricht):
                \begin{description}
                    \item[85] Erziehung und Unterricht 
                    \item[85.5] Sonstiger Unterricht
                    \item[85.59] Sonstiger Unterricht a.n.g.\footnote{``anderweitig nicht genannt''}
                    \item[85.59.2] Berufliche Erwachsenenbildung
                \end{description}
        \end{itemize}
      
    \end{frame}
    \begin{frame}{Nationalitätenschlüssel}
        \begin{itemize}
            \item veröffentlicht von destatis (in internationaler Abstimmung)
            \item enthält:
            \begin{itemize}
                \item Staatsnamen in Kurzform (``Vatikanstadt''),
                \item \dots und in Vollform (``Staat Vatikanstadt''),
                \item Staatsangehörigkeit (``vatikanisch''),
                \item  Kontinent (``EUR''),
                \item zwei- und dreistelliger ISO-3166-1-Code (`VA''\footnote{Genutzt insb. als Top Level Domain (TLD) im WWW}, ``VAT''),
                \item Destatis-BEV-Code (``167'')\footnote{Dieser BEV-Code entspricht dem in der Bevölkerungsstatistik verwendeten Nationalitätenschlüssel im engeren Sinn, nicht zu verwechseln mit anderen Schlüsseln, die für den Internationalen Handel verwendet werden.} 
            \end{itemize}    
        \end{itemize}
    \end{frame}
    \begin{frame}{Aufbau des BEV-Code}
        \begin{description}
            \item[000:] Deutschland
            \item[1xx:] Europa (ohne D)
            \item[2xx:] Afrika
            \item[3xx:] Amerika(s)
            \item[4xx:] Asien
            \item[5xx:] Australien und Ozeanien
            \item[unterhalb der ``kontinentalen''-Ebene:] grobe alphabetische Reihenfolge
            \end{description}
    \end{frame}
    \begin{frame}{BEV-Code-Problem an Beispiel Jugoslawien}
        \begin{table}
    \centering
        \begin{tabular}{llll}
            \textbf{Staat}  & \textbf{Code} & \textbf{gültig von} & \textbf{gültig bis}\\
                Bosnien und Herzegowina & 122 & 01.03.1992 & \dots             \\
                Jugoslawien & 120 & \dots & 26.04.1992      \\
                Bundesrepublik Jugoslawien & 138 & 27.04.1992 & 04.02.2003      \\
                Kosovo & 150 & 17.02.2008 & \dots \\
                Kroatien & 130 & 25.06.1991 & \dots \\
                Montenegro & 141 & 03.06.2006 & \dots \\
                Nordmazedonien & 144 & 08.09.1991 & \dots \\
                Serbien und Montenegro & 132 & 05.02.2003 & 02.06.2006 \\ 
                Serbien (mit Kosovo) & 133 & 03.06.2006 & 16.02.2008 \\
                Serbien & 133 & 17.02.2008 & \dots \\
                Slowenien & 131 & 25.06.1991 & \dots \\
        \end{tabular}
    \end{table}
    \end{frame}
    
    \begin{frame}{GV-ISys}
        \begin{block}{Gemeindeverzeichnis-Informationssystem}
            \begin{itemize}
                \item veröffentlicht von destatis
                \item enthält:
                    \begin{itemize}
                        \item Amtlicher Regionalschlüssel (ARS)
                        \item Amtlicher Gemeindeschlüssel (AGS)
                        \item PLZ des Verwaltungssitzes
                        \item Fläche in km\textsuperscript{2}
                        \item Einwohnerzahl (insgesamt/männlich/weiblich)
                        \item siedlungsstrukturelle Typisierungen
                    \end{itemize}
            \end{itemize}
        \end{block}
    \end{frame}
    