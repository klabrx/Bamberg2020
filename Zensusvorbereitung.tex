\section{Zensusvorbereitung}
    \begin{frame}{Verantwortung für den Zensus 2021/22}
        \begin{itemize}
            \item Die Verantwortung für den Zensus liegt, wie schon 2011, beim Statistischen Bundesamt.
            \item Die Durchführung vor Ort obliegt dagegen den Kommunen und Kreisen.
            \item Somit müssen die Länder und Landesämter als Schnittstelle zwischen Bund und Kommunalebene fungieren (Föderalismusprinzip).
            \item Da jedoch die kommunalen Spitzenverbände ebenfalls an der Vorbereitung beteiligt sind (insbesondere im Bereich der Konnexitätsverhandlungen), sitzen auch einzelne Städte mit am großen Tisch.
        \end{itemize}
    \end{frame}
    \begin{frame}{Kostenerstattung}
        \begin{itemize}
            \item Der Aufwand für die Vor-Ort-Aufgaben des Zensus muss den Kommunen erstattet werden.
            \item Diese Konnexitätsverhandlungen führten beim letzten Zensus zu erheblichem Konfliktpotential.
            \item Durch die Einbindung der Kommunen schon im Vorfeld\footnote{Leider nicht ab dem Zeitpunkt und in dem Umfang, in dem die Kommunen dies für sinnvoll gehalten hätten) soll vermieden werden, dass große Überaschungen bei der Einschätzung des anfallenden Arbeitsaufwandes entstehen.}
            \item Die Zeit wird zeigen, ob das so klappt.
        \end{itemize}
    \end{frame}
    \begin{frame}{Software, Abläufe}
        \begin{itemize}
            \item Im Gegensatz zu 2011 wird die Zensussoftware diesmal direkt unter Federführung von Destatis entwickelt.
            \item Kommunen aus dem ganzen Bundesgebiet nehmen regelmäßig an sog- ``Show and Tell''-Veranstaltungen teil, in denen Zwischenstände der Software vorgestellt, besprochen und auch kritisiert werden.
            \item Einwände und Anregungen der Kommunen werden dabei durchaus berücksichtigt (wenn auch nicht alle).
        \end{itemize}
    \end{frame}
    \begin{frame}{Verschiebung des Zensus um ein Jahr}
        \begin{itemize}
            \item Die Corona-Pandemie hat Anfang 2020 die laufenden Vorbereitungsarbeiten für den Zensus praktisch zum Erliegen gebracht.
            \item Der dadurch entstehende Zeitverlust ist bis zum Zensusstichtag nicht mehr aufzuholen.
            \item Kommunen wurden informiert, keine wirtschaftlichen Dispositionen (Personal, Räumlichkeiten) für 2021 mehr vorzunehmen.
            \item Die endgültige Entscheidung aus Brüssel steht jedoch noch aus.
        \end{itemize}
        
    \end{frame}