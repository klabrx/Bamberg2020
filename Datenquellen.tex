\section[Datenquellen]{Datenquellen}
\subsection[Kategorisierung]{Kategorisierung}
\begin{frame}[allowframebreaks]{Primäre vs. Sekundäre Daten}
    \begin{description}
        \item[Primärdaten] werden eigens für die jeweilige Fragestellung erhoben.
        \begin{itemize}
            \item Typische Erhebungsform: Befragung oder Beobachtung
            \item Beispiele: Bürgerbefragung, Mietspiegelbefragung, Verkehrsfrequenzmessung
            \item Primärstatistische Stichprobenerhebungen erfordern ausgeprägte Methodenkenntnisse
            \item Vorteil: Passgenau zur Fragestellung
            \item Nachteil: Aufwendig und teuer
        \end{itemize}
        \framebreak
        \item[Sekundärdaten] ``liegen bereits vor'', z.B. in Informationssystemen oder Datensammlungen
        \begin{itemize}
            \item Beispiele: Arbeitsmarktzahlen, Geschäftsstatistiken, Melderegister, Besucherzahlen
            \item Nachteil: Erheblich günstiger zu bekommen
            \item Nachteil: normalerweise ``gekauft wie gesehen und probegefahren'', d.h. weder die Gewinnung noch die Vorab-Aufbereitung ist in größerem Umfang steuerbar.
        \end{itemize}
        \framebreak
        \item[Grundsatz:] Vor jeder Erhebung (Primärdatengewinnung) muss geklärt werden, ob nicht schon Sekundärdaten zur Verfügung stehen.
    \end{description}
\end{frame}
\begin{frame}[allowframebreaks]{Individual- vs. Aggregatdaten}
    \begin{description}
        \item[Individualdaten, bzw. Einzeldaten] haben Individuen als Merkmalsträger
            \begin{itemize}
                \item typische Merkmalsträger in der Kommunalstatistik: Personen, Kraftfahrzeuge, Unternehmen, Wohnungen
                \item Herkunft entweder aus Verwaltungsregistern oder aus Befragungen/Erhebungen
                \item Regelmäßig Grund für gründliche datenschutzrechtliche Begleitung
                \item Vorteil: Uneingeschränkter und ungefilterter Informationsgehalt für statistische Analysen, hohe Flexibilität bei Aggregation und Verknüpfung
                \item Beispiele: Ergebnisse einer Bürgerbefragung, Datenabzüge aus dem Melderegister
            \end{itemize}
            \framebreak
        \item[Aggregatdaten] entstehen aus der Verarbeitung, insb. Aggregation von Einzeldaten zu größeren Beobachtungseinheiten
            \begin{itemize}
                \item Beispiele: Aggregierung von Einwohnerzahlen auf Stadtteilebene, Wahlergebnisse auf Stimmkreisebene
                \item Hauptanwendung: Basis für Berichtswesen und statistische Informationssysteme
                \item Vorteil: Einfachere und schnellere Verarbeitbarkeit, höhere Benutzerfreundlichkeit, bei passender Aufbereitung datenschutzrechtlich unbedenklich
                \item Nachteil: Mit jeder Aggregationsstufe gehen Detailinformationen verloren, die Eignung als Datenquelle für weitere Auswertungen nimmt rapide ab.
                
            
            \end{itemize}
    \end{description}
\end{frame}