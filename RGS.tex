\section{Kleinräumige Gliederung}
\begin{frame}[allowframebreak]{Kleinräumige Gliederung}
    \begin{exampleblock}{Alles, was passiert, passiert irgendwo.}
        Räumliche Gliederungssysteme erlauben die Zuordnung von Beobachtungen, Ereignissen, Erhebungsergebnissen zu geografischen Einheiten.
    \end{exampleblock}
    Drei Gliederungssysteme sind in der Kommunalstatistik relevant:
    \begin{description}
        \item[Nationalitätenschlüssel] steht für eine rechtliche Staatsangehörigkeit, ersatzweise auch für den Migrationshintergrund. Dieser 3-stellige sog. "BEV-Code"\cite{bevcode} wird von statistischen Bundesamt (destatis) regelmäßig aktualisiert und veröffentlicht.
        \item[Allgemeiner Gebietsschlüssel]\cite{ags} bezeichnet die Zugehörigkeit zu einer Gemeinde. Der hierarchische Aufbau erlaubt auch die Zuordnung einer Gemeinde (am Beispiel Bayern) zum Kreis, Bezirk und Bundesland.
        \framebreak
        \item[Kleinräumige Gliederung]\cite{agk} steht für eine untergemeindliche Gebietseinteilung und Stadtteile, Bezirke, Distrikte, Reviere, Kieze \dots
    \end{description}
    
\end{frame}