\section[Abschottung]{Abschottung}
\begin{frame}{Auf EU-Ebene: EU-DSGVO oder EU-StatV}
\metroset{block=fill}
    \begin{block}{Abschottung}
        Der Begriff taucht kein einziges mal explizit auf.
    \end{block}
    \begin{block}{Art. 89 Abs. 1 DSGVO}
        ``Die Verarbeitung \dots zu statistischen Zwecken unterliegt geeigneten Garantien \dots. Mit diesen Garantien wird sichergestellt, dass technische und organisatorische Maßnahmen bestehen, mit denen insbesondere die Achtung des Grundsatzes der Datenminimierung gewährleistet wird.''
    \end{block}
    \begin{block}{Art. 20 EU-StatV}
        ``Vertrauliche Daten, die ausschliesslich für die Erstellung europäischer Statistiken erhoben wurden, werden von den NSÄ und anderen einzelstaatlichen Stellen und von der Kommission (Eurostat) ausschliesslich für statistische Zwecke verwendet, (\dots)''
    \end{block}

\end{frame}

\begin{frame}{Auf Bundesebene: BDSG oder BStatG?}
\metroset{block=fill}
    \begin{block}{Abschottung}
        Der Begriff taucht kein einziges mal explizit auf.
    \end{block}
    \begin{block}{\S50 BDSG}
        Garantien, die ``\dots in einer räumlich und organisatorisch von den sonstigen Fachaufgaben getrennten Verarbeitung bestehen.''
    \end{block}
    \begin{block}{\S16 Abs. 5 Satz 2 BStatG: Übermittlung nur, wenn}
        ``\dots wenn durch Landesgesetz eine Trennung dieser Stellen von anderen kommunalen Verwaltungsstellen sichergestellt und das Statistikgeheimnis durch Organisation und Verfahren gewährleistet ist.''
    \end{block}    
\end{frame}

\begin{frame}{Auf Länderebene: BayDSG oder BayStatG}
\metroset{block=fill}
    \begin{block}{Abschottung}
        Der Begriff taucht kein einziges mal explizit auf. Insb. das BayDSG geht auf Statistikstellen nicht gesondert ein.  
    \end{block}
    \begin{block}{Art. 20 Abs. 2 Satz 2 BayStatG}
        ``Statistikstellen\footnote{Anm.: im übertragenen Wirkungskreis, d.h. im staatlichen Auftrag} sind räumlich und organisatorisch von anderen Verwaltungsstellen zu trennen, gegen den Zutritt unbefugter Personen hinreichend zu sichern und mit Personal auszustatten, das die Gewähr für Zuverlässigkeit und Verschwiegenheit bietet.''
    \end{block}
    \begin{block}{Art. 24 Abs. 2 Satz 2 BayStatG}
        ``\textsuperscript{1}Statistikstellen\footnote{Anm.: im eigenen Wirkungskreis, d.h. im Rahmen der Selbstverwaltung} sind durch Satzung einzurichten, die auch die wesentlichen organisatorischen Bestimmungen, vornehmlich zur Wahrung des Statistikgeheimnisses zu treffen hat. \textsuperscript{1}Art. 20 Abs. 2 und 3 finden entsprechende Anwendung.''
    \end{block}    
\end{frame}