\section[Abschottung]{Abschottung}
\begin{frame}[allowframebreaks]{Abschottung}
\begin{description}
    \item [Abschottung] taucht als Begriff in vielen gesetzlichen Regelungen aller vier Ebenen auf, eine konkrete Definition ist jedoch kaum zu finden.
    \item [Art. 89 Abs. 1 DSGVO] Die Verarbeitung \dots zu statistischen Zwecken unterliegt geeigneten Garantien \dots. Mit diesen Garantien wird sichergestellt, dass technische und organisatorische Maßnahmen bestehen, mit denen insbesondere die Achtung des Grundsatzes der Datenminimierung gewährleistet wird.
    \item[Art. 89 Abs. 2] Werden personenbezogene Daten \dots zu statistischen Zwecken verarbeitet, können \dots im Recht der Mitgliedstaaten insoweit Ausnahmen von den Rechten gemäß der Artikel 15, 16, 18 und 21 vorgesehen werden, als diese Rechte voraussichtlich die Verwirklichung der spezifischen Zwecke unmöglich machen oder ernsthaft beeinträchtigen und solche Ausnahmen für die Erfüllung dieser Zwecke notwendig sind.
    \item[Art. 15] \dots das Recht auf Auskunft über diese personenbezogenen Daten \dots
    \item[Art. 16] \dots das Recht, \dots die Berichtigung sie betreffender unrichtiger personenbezogener Daten \dots (und) \dots die Vervollständigung unvollständiger personenbezogener Daten \dots zu verlangen.
    \item[Art. 18]\dots das Recht, \dots die Einschränkung der Verarbeitung zu verlangen, \dots'
    \item[Art. 21]\dots das Recht, \dots gegen die Verarbeitung sie betreffender personenbezogener Daten, \dots Widerspruch einzulegen'
\end{description}
\end{frame}