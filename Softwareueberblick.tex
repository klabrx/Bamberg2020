\section{Softwareoptionen}
    \begin{frame}{Standardsoftware}
        \begin{itemize}
            \item Office-Pakete sind allgegenwärtig, für den Produktiveinsatz grundsätzlich auch geeignet. Limits bei der Anzahl der Zeilen gibt es eigentlich nicht mehr. Gestaltungsmöglichkeiten für Berichte, Tabellen und Diagramme sind durchaus brauchbar.
            \item An DB-Software (Oracle, MS-SQL, mySQL, \dots) führt eigentlich kein Weg vorbei.
        \end{itemize}
    \end{frame}
    \begin{frame}{Spezielle Statistiksoftware}
        \begin{itemize}
            \item SPSS (bzw. der OpenSource-Clon PSPP) sind in vielen Städten im Einsatz, jedoch in nicht unerheblichem Ausmaß kostenpflichtig
            \item R scheint sich in den letzten Jahren an den Hochschulen immer mehr durchzusetzen. Seit 2019 gibt es im VDSt die Kosis-Gemeinschaft ``Ko-R'', die zum Ziel hat, R als gemeinsame Entwicklungs- und Anwendungssprache zu etablieren.
            \item SAS
            \item Stata
            \item Statistica
        \end{itemize}
    \end{frame}
    \begin{frame}{Kosis-Software}
        \begin{description}
            \item[AGK\footnote{Das ``AG'' in ``AGK'' steht \textbf{nicht} für ``Andreas Gleich''. Dahingehende Gerüchte sind nur leicht übertrieben.}] \textbf{A}dressen, \textbf{G}ebäude, \textbf{K}leinräumige Gliederung
            \item[HHSTAT] Haushaltsstrukturen aus den Informationen der Melderegister
            \item[DUVA] Informationsmanagement
            \item[Kosis-APP] Kleinräumige Statistikdaten für unterwegs
            \item[KO-Umfrage] Organisation und Durchführung von Umfragen
            \item[KO-Wahl] Wahlanalyse
            \item[Sikurs] Kleinräumige Bevölkerungsprognose
            \item[KO.R] Analyse- und Auswertungstools mit R
        \end{description}
    \end{frame}
    \begin{frame}{HHSTAT}
        \includegraphics[width=150pt]{images/HHSTAT.png}
        \centering
        \begin{table}[]
            \centering
            \begin{tabular}{lp{9cm}}
                 \textbf{EwoPEaK:} & Protokollierte(!) Plausibilisierung, Korrektur, Ergänzung und Editierung von Einwohnerdateien (Bewegung und Bestand) \\
                 \textbf{Migrapro:} & Ableitung von Migrationshintergründen \\
                 \textbf{HHGEN:} &  Generierung von Haushalten aus dem Melderegister \\
                 \textbf{Gizeh:} & Konfigurierbare Pyramidendiagramme \\
            \end{tabular}
        \end{table}
    \end{frame}
    \begin{frame}{Duva}
        \includegraphics[width=150pt]{images/DUVA.png}
        \centering
        \begin{table}[]
            \centering
            \begin{tabular}{lp{9cm}}
                 \textbf{NWS:} & Metadatenbasiertes Nachweissystem, enthält Dateiobjekte, deren Satzbeschreibungen, Codierungen, Referenztabellen, \dots  \\
                 \textbf{FormGen:} & Formulargenerator für On- und Offline-Datenerfassung \\
                 \textbf{IERF:} & Datenerfassung am PC oder im Web \\
                 \textbf{ASW:} & Auswertungsassistent zur flexiblen Gestaltung von Tabellen und Diagrammen \\
                 \textbf{Info-Portal:} & Aufbau und Pflege eines Informationsangebot \\
            \end{tabular}
        \end{table}
    \end{frame}
    \begin{frame}{Kosis-APP}
        \includegraphics[width=150pt]{images/KosisApp.png}
        \centering
        \begin{itemize}
            \item Arbeitslose und Beschäftigte nach Geschlecht
            \item Einwohner nach Altersgruppen, Familienstand, Migrationshintergrund
            \item Geburten, Sterbefälle und Wanderungen
            \item Haushalte nach Haushaltsgröße
            \item Wohnungen nach Anzahl der Räume
            \item Zweitstimmenanteile bei der Bundestagswahl
        \end{itemize}
    \end{frame}
    \begin{frame}{KO-Umfrage}
        \includegraphics[width=150pt]{images/KOUmfrage.png}
        \centering
        \begin{itemize}
            \item Keine eigene Software-Entwicklung
            \item ``Softwareanwendergemeinschaft'', Schwerpunkt derzeit ``Blubbsoft''
        \end{itemize}
    \end{frame}
    \begin{frame}{KO-Wahl}
        \includegraphics[width=150pt]{images/KOWahl.png}
        \centering
        \begin{itemize}
            \item \textbf{Keine} eigene Software-Entwicklung
            \item IT-Fachverfahren zur Wahlorganisation und Ergebnisermittlung
            \item Berechnung von Wählerwanderungen
            \item Erhebungsinstrumente und Organisationshilfen für Wahltagsbefragungen
            \item Wandel im Verhältnis von Briefwahl und Lokalwahl
            \item Verfahren zur Hochburgenanalyse
            \item Umrechnung von Wahlergebnissen (Gebietsstandänderungen, Einrechnung von Briefwahlergebnissen in allgemeine Wahlbezirke
        \end{itemize}
    \end{frame}
    \begin{frame}{Sikurs}
        \includegraphics[width=150pt]{images/Sikurs.png}
        \centering
        \begin{itemize}
            \item Kleinräumige Bevölkerungsprognose
            \item Baukastenprinzip aus 17 Bausteinen
                \begin{itemize}
                    \item AUßenwanderung
                    \item Binnenwanderung
                    \item Neubaubezug
                    \item \dots
                \end{itemize}
            \item Unterstützung verschiedener Szenarien
            \item Sikurs-Modell wird auch von Landesämtern eingesetzt
        \end{itemize}
    \end{frame}
    \begin{frame}{KO.R}
        \includegraphics[width=100pt]{images/KoR.png}
        \centering
            \begin{itemize}
                \item Gemeinsame Entwicklung und Anwendung von Methoden zur Datenauswertung mit \includegraphics[height=\baselineskip]{images/Rlogo.png}
                \item Ausbau von Statistik-Packages in \includegraphics[height=\baselineskip]{images/Rlogo.png} für kommunalstatistische Anwendungen
                \item Gemeinsame Planung und Durchführung von Plausibilisierungen von Datenbeständen
                \item Automatisierungen im Reporting und die Gestaltung von Dashboards
                \item Aufbau und Pflege einer gemeinsamen Informations- und Austauschplattform
            \end{itemize}
    \end{frame}