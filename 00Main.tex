\documentclass[10pt]{beamer}
\usepackage[utf8]{inputenc}
\usepackage{hyperref}
\usepackage[greek,ngerman]{babel}
\usetheme[progressbar=frametitle]{metropolis}
\usecolortheme{seagull}


% \usepackage{appendixnumberbeamer}
% resets frame number counter to 0 when reaching the \appendix command

% \setbeamertemplate{frametitle continuation}{\insertcontinuationcount}

% \usepackage{booktabs}
% Erlaubt das Einfügen von Tabellen in druckreifem Layout

% \usepackage{pgfplots}
% \usepgfplotslibrary{dateplot}
% für hochwertige Funktionsplots

% \usepackage{xspace}
% \newcommand{\themename}{\textbf{\textsc{metropolis}}\xspace}
% Zur Darstellung des Themennamens in Fettdruck

\usepackage{graphicx}
\graphicspath{ {./images/} }

\usepackage{xpatch} %einfacheres Patchen von Kommandos

\makeatletter
\patchcmd\beamer@@tmpl@frametitle{\insertframetitle}{\insertsection: \insertframetitle}{}{}
\makeatother


\usepackage[backend=biber,style=alphabetic,sorting=ynt]{biblatex}
\addbibresource{Verweise.bib}


\title{Grundlagen der Kommunalstatistik }
\subtitle{Einsteigerseminar }
\date{\today}
\date{}
\author{Klaus Brückner}
\institute{Statistikstelle der Stadt Passau}
\date{Herbst 2020}
% \titlegraphic{\hfill\includegraphics[height=1.5cm]{logo.pdf}}

\setbeamertemplate{frame footer}{Grundlagen der Kommunalstatistk - Bamberg, Herbst 2020}
\setbeamertemplate{frametitle continuation}{\insertcontinuationcount}
\setbeamerfont{section in toc}{size=\small}
\setbeamerfont{subsection in toc}{size=\small}

\begin{document}

\maketitle

\section*{Einleitung}
\begin{frame}[allowframebreaks]{Outline}
  \setbeamertemplate{section in toc}[sections numbered]
  \tableofcontents[hideallsubsections]
\end{frame}

\section[Einführung]{Einführung}

\begin{frame}[fragile]{Mit wem haben Sie es zu tun?}
    \begin{itemize}
        \item Klaus Brückner
        \begin{itemize}
            \item Leiter und gleichzeitig einziger Mitarbeiter der Statistikstelle der Stadt Passau
            \item \emph{kein} gelernter Statistiker oder Geograph
            \item Zensus-Erhebungsstellenleiter 2011 und wohl auch 2021/22
            \item nebenher zuständig für Breitbandausbau und öffentliches WLAN
        \end{itemize}
        \item Schwerpunkte der statistischen Arbeit
        \begin{itemize}
            \item Demografiebeobachtung (z.B. für Kita-Planung)
            \item Qualifizierter Mietspiegel
            \item Automatisierte Verkehrszählung
        \end{itemize}
        \item Zusatzaufgaben, "Nebenjobs"
        \begin{itemize}
            \item Breitbandausbau
            \item Öffentliches WLAN
        \end{itemize}
        
    \end{itemize}
  
\end{frame}
\section{Ebenen der amtlichen Statistik}
    \begin{frame}{EU}
   \metroset{block=fill}
        \begin{block}{Eurostat}
            \begin{description}
                \item[Sitz:] Luxemburg
                \item[Rang:] Generaldirektion der Europäischen Kommission
                \item[Aufgabe:] Standardisierung, Verarbeitung und Veröffentlichung vergleichbarer Daten auf EU-Ebene
                \item[Arbeitsweise:] führt selbst \alert{keine} Erhebungen durch, macht Vorgaben an die Mitgliedsländer
            \end{description}
        \end{block}
   \end{frame}
    \begin{frame}{Bund}
    \metroset{block=fill}
        \begin{block}{Statistisches Bundesamt - destatis}
            \begin{description}
                \item[Sitz:] Wiesbaden
                \item[Rang:] Bundesamt, dem BMI unterstellt
                \item[Aufgabe:] siehe \S1 BStatG
                \item[veröffentlicht] verbindliche Klassifikationen, Verzeichnisse, Systematiken, z.B.
                    \begin{itemize}
                        \item Güterverzeichnis für Produktionsstatistiken (GP2019)
                        \item Klassifikation der Wirtschaftszweige (WZ2008)
                        \item Klassifikation der Berufe (KldB2010)
                        \item Staats- und Gebietssystematik
                        \item Gemeindeverzeichnis-Informationssystem GV-ISys (aka ``AGS'')
                    \end{itemize}
            \end{description}
        \end{block}
    \end{frame}
    \begin{frame}[allowframebreaks]{Länder}
    \metroset{block=fill}
        \begin{block}{Statistische Landesämter}
            \begin{itemize}
                \item Baden-Württemberg (Stuttgart)
                \item Bayern (Fürth)
                \item Berlin-Brandenburg (Potsdam)
                \item Bremen (Bremen)
                \item Hamburg und Schleswig-Holstein (``Statistik Nord'' in Hamburg und Kiel)
                \item Hessen (Wiesbaden)
                \item Mecklenburg-Vorpommern (Schwerin)
            \end{itemize}
        \end{block}            
    \framebreak
        \begin{block}{Statistische Landesämter (cont.)}
            \begin{itemize}            
                \item Niedersachsen: (Hannover)
                \item Nordrhein-Westfalen (``Landesbetrieb IT-NRW'' in Düsseldorf)
                \item Rheinland-Pfalz (Bad Ems)
                \item Saarland (Saarbrücken)
                \item Sachsen (Kamenz)
                \item Sachsen-Anhalt (Halle an der Saale)
                \item Thüringen (Erfurt)
            \end{itemize}
        \end{block}
    \framebreak
        \begin{block}{Profil der StatLÄ}
            \begin{description}
                \item[Rang:] Im Allgemeinen eine Oberste Landesbehörde
                \item[Aufgabe:] Im jeweiligen Landesstatikgesetz\footnote{z.B. \S2 Abs. 1 Satz 1 LStatG NRW: Die Landesstatistik hat die Aufgabe, Daten zu erheben, zu sammeln, aufzubereiten, darzustellen und zu analysieren, soweit Landesrecht dies bestimmt.} geregelt
                \item[Schnittstelle:] Da der Bund Aufgaben nicht direkt an die Gemeinden übertragen kann, agieren die StatLÄ als Mittler, z.B. beim Zensus
            \end{description}
        \end{block}
    \end{frame}
    \begin{frame}{Kreise und Kommunen}
    \metroset{block=fill}
        \begin{block}{Organisationsform}
            von der kleinen Einpersonenstelle bis hin zum großen, dem Direktorium unmittelbar unterstellten statistischen Amt ...
        \end{block}
        \begin{block}{Rang, Hierarchie}
            Stabstelle, Fachabteilung, Amt, Referat, Fachgruppe, ... 
        \end{block}        
        \begin{block}{Aufgabe}
            in den jeweiligen Satzungen und Dienstanweisungen geregelt
        \end{block}    
        \begin{block}{Alleinstellungsmerkmale}
            \begin{itemize}
                \item Auswertungen in kleinräumiger, untergemeindlicher Gliederung
                \item Detaillierte Ortskenntnis\footnote{Wo die Ortskenntnis der Landesämter mikroskopisch klein ist, ist die der Kommunalstatistik mikroskopisch genau.}
                \item Kurze Wege zu kommunalen Datenquellen
            \end{itemize}
           
            
        \end{block}    
  
    \end{frame}
\section{Rechtsrahmen}
% \begin{frame}{Inhalt}
%     \tableofcontents[currentsection, hideothersubsections]
% \end{frame}



\subsection{Rechtliche Ebenen}
\begin{frame}[fragile]{Rechtliche Ebenen}
    \metroset{block=fill}
    \begin{block}{Europarecht: EU-Statistikverordnung}
    (und Regelungen mit Statistikbezug in anderen Verordnungen, wie z.B. EU-DSGVO, Durchführungsverordnungen zum Zensus, ...)
        \begin{block}{Bundesrecht: Bundesstatistikgesetz}
        (und Regelungen mit Statistikbezug in anderen Bundesgesetzen, z.B. zum Zensus, SGB, ...)

            \begin{block}{Landesrecht: Statistikgesetze der Länder, z.B. BayStatG}
            (und Regelungen mit Statistikbezug in anderen Landesgesetzen, z.B. zum Zensus)
                \begin{block}{Satzungen und Verordnungen auf kommunaler Ebene}
                (z.B. Mietspiegel-Erhebungssatzung)
                \end{block}
            \end{block}
        \end{block}
    \end{block}
\end{frame}

\subsection{Die Statistikgesetze im Einzelnen}
\begin{frame}{EU-Statistikverordnung}
    \metroset{block=fill}
    \begin{block}{``Hausgesetz'' von Eurostat}
        \begin{description}
            \item[Bedeutung für die Kommunalstatistik:] Nicht unmittelbar, nur auf dem Umweg über Bundes- und Landesstatistik
            \item[Grundsätzliches:] Die EU-StatV enthält viele Definitionen, formuliert Grundsätze und Verhaltensregeln, wie sie in ähnlicher Form auch in Bundes- und Landesgesetzen stehen.
        \end{description}
    \end{block}
\end{frame}

\begin{frame}{Bundesstatistikgesetz}
\metroset{block=fill}
    \begin{block}{Föderalismusprinzip}
``Durchregieren'' vom Bund zu den Kommunen ist im Föderalismusprinzip nicht vorgesehen. Kommunale Statistische Ämter arbeiten nicht unmittelbar für den Bund, deshalb ist das BStatG auch nicht von unmittelbarer Bedeutung.
\end{block}
    \begin{block}{ABER: Rahmen für die Landesgesetze}
        Das BStatG enthält eine Reihe von Grundsätzen, Begriffsbestimmungen, \dots, die ebenso in den verschiedenen Landesgesetzen stehen könnten (und zum Teil auch dort stehen).
    \end{block}
\end{frame}

\begin{frame}{Landesstatistikgesetze}
    \metroset{block=fill}
    \begin{block}{Landes-/Bundes- vs. Kommunalstatistiken} Die Landesstatistikgesetze regeln insbesondere auch Art und Umfang der Tätigkeit kommunaler Stellen im staatlichen Auftrag (``übertragener Wirkungskreis'').
    \end{block}
    \begin{block}{Erhebungsstellen}
    EHst werden häufig zweimal beschrieben:
            \begin{itemize}
                \item Art. 20 BayStatG oder ThürStatG: Statistikstellen im staatlichen Auftrag
                \item Art. 24 BayStatG oder ThürStatG: Statistkstellen im Rahmen der kommunalen Selbstverwaltung
            \end{itemize}
    \end{block}
    \begin{block}{Einrichtung durch Satzung}
    EHSt-Einrichtung durch Satzung wird in den Landesstatistikgesetzen häufig vorgeschrieben.
    \end{block}

\end{frame}

\begin{frame}{Satzungen auf kommunaler Ebene}
    \metroset{block=fill}
    \begin{block}{Konstituierende Satzungen}
        z.B. aus Art. 24 Satz 2 Satz 1 BayStatG ergibt sich sowohl Verpflichtung als auch Ermächtigung für den Erlass der Statistiksatzung der Stadt Passau.
    \end{block}
    \begin{block}{Projektbezogene Satzungen, z.B.}
        \begin{itemize}
            \item Mietspiegelerhebungssatzung
            \item Bürgerbefragungssatzung
        \end{itemize}
    \end{block}
\end{frame}
\begin{frame}{Weitere Regelungen}
    \metroset{block=fill}
    \begin{block}{Dienstanweisungen, Betriebskonzepte}
        \begin{itemize}
            \item Als rein organisatorische Richtlinien eigentlich nicht dem Rechtsrahmen zuzurechnen
            \item Konkretisierung von Anforderungen z.B. aus Art. 89 EU-DSGVO (``\dots geeignete Garantien für Rechte und Freiheiten der betroffenen Person\dots'')
        \end{itemize}
    \end{block}
    

\end{frame}

\section{Datenschutz durch Abschottung}
    \begin{frame}{EU}
    \metroset{block=fill}
        \begin{block}{Art. 89 Abs. 1 EU-DSGVO (``Garantenstellung''}
            ´´(\dots) Mit diesen Garantien wird sichergestellt, dass technische und organisatorische Maßnahmen bestehen, mit denen insbesondere die Achtung des Grundsatzes der Datenminimierung gewährleistet wird. Zu diesen Maßnahmen kann die Pseudonymisierung gehören, sofern es möglich ist, diese Zwecke auf diese Weise zu erfüllen. (\dots)''
        \end{block}
        % \begin{block}{Bundesrecht, Landesrecht, Kommunales Recht}
        %     Vergleichbare Regelungen finden sich auf jeder Ebene.
        % \end{block}
    
    \end{frame}
    \begin{frame}{Bund}
    \metroset{block=fill}
        \begin{block}{Abschottung}
            Der Begriff taucht kein einziges mal explizit auf.
        \end{block}
        \begin{block}{\S50 BDSG}
            Garantien, die ``\dots in einer räumlich und organisatorisch von den sonstigen Fachaufgaben getrennten Verarbeitung bestehen.''
        \end{block}
        \begin{block}{\S16 Abs. 5 Satz 2 BStatG: Übermittlung nur, wenn}
            ``\dots wenn durch Landesgesetz eine Trennung dieser Stellen von anderen kommunalen Verwaltungsstellen sichergestellt und das Statistikgeheimnis durch Organisation und Verfahren gewährleistet ist.''
        \end{block}    
    \end{frame}

    \begin{frame}{Länder}
    \metroset{block=fill}
        \begin{block}{Art. 20 Abs. 2 Satz 2 BayStatG}
            ``Statistikstellen\footnote{Anm.: im übertragenen Wirkungskreis, d.h. im staatlichen Auftrag} sind räumlich und organisatorisch von anderen Verwaltungsstellen zu trennen, gegen den Zutritt unbefugter Personen hinreichend zu sichern und mit Personal auszustatten, das die Gewähr für Zuverlässigkeit und Verschwiegenheit bietet.''
        \end{block}
        \begin{block}{Art. 24 Abs. 2 Satz 2 BayStatG}
            ``\textsuperscript{1}Statistikstellen\footnote{Anm.: im eigenen Wirkungskreis, d.h. im Rahmen der Selbstverwaltung} sind durch Satzung einzurichten, die auch die wesentlichen organisatorischen Bestimmungen, vornehmlich zur Wahrung des Statistikgeheimnisses zu treffen hat. \textsuperscript{1}Art. 20 Abs. 2 und 3 finden entsprechende Anwendung.''
        \end{block}    
    \end{frame}
    \subsection{Räumliche Abschottung}
        \begin{frame}{Bauliche Maßnahmen}
            \begin{block}{Mindestanforderung}
            \begin{itemize}
                \item Räumlich getrennt vom restlichen Verwaltungsvollzug
                \item Getrennt abschließbar mit dokumentiertem und restriktivem Schlüsselregime
                \item Einbruchs-/vandalismussicher
                \item nicht von außen einsehbar
            \end{itemize}
        \end{block}
            \begin{block}{Wünschenswert}
                \begin{itemize}
                \item Türöffner, Sprechanlage (mit Kamera)
                \item Zugang zu vergleichbar sicheren Archivräumen (je nach Bedarf)
            \end{itemize}
            \end{block}
        \end{frame}
    \subsection{Organisatorische Abschottung}
        \begin{frame}{Hierarchie}
            \begin{block}{Unabhängig und weisungsungebunden}
                
            \end{block}
        \end{frame}{}
    
\section{Aufgabenfelder}
    \begin{frame}[allowframebreaks]{Aus den Stellenausschreibungen}
        \begin{itemize}
            \item Durchführung uns Auswertung kommunaler Erhebungen
            \item Erarbeitung von Veröffentlichungen zum Aufgabengebiet
            \item Entwicklung von relationalen und dimensionalen Datenmodellen
            \item Erstellung von Teilbereichen des Mietspiegels
            \item Beraten der Fachverwaltung in Fragen des Aufbaus von Datenbeständen und des Datenmanagements
            \item Thematisch breit aufgestellte Beobachtung des Stadtraums und der sich verändernden Entwicklungsbedingungen sowie die frühzeitige Wahrnehmung von Trends und deren Bewertung
            \item Erstellung und Interpretation von Prognosen
            \item Raumbezugssystem
            \item Geografisches Informationssystem
            \item Datengewinnung
                \begin{itemize}
                    \item zur Bevölkerungsstruktur
                    \item zum Wohnungsbestand
                    \item zu Beschäftigten und Arbeitslosen
                    \item zu Empfängern von Leistungen
                    \item zu sozialen und kulturellen Einrichtungen
                \end{itemize}
            \item Wahlanalyse und Mitarbeit bei der Wahlorganisation
            \item Plausibilisierung und Qualitätssicherung von Registerdaten
            \item Erstellung kleinräumiger Gebietsprofile
            \item (\dots)
        \end{itemize}
    \end{frame}
    
    
\section{Datenlogistik}

  \begin{frame}{Daten -> Informationen -> Wissen}
  \metroset{block=fill}
    \begin{block}{Daten}
      Daten sind zunächst kontextlose Folgen von Zeichen, Ziffern, Hieroglyphen, Symbolen, Emojies ...
    \end{block}
    \begin{block}{Informationen}
      Werden Daten in einen Kontext\footnote{Dieser Kontext kann aus einer Satzbeschreibung bestehen.} gestellt oder lösen sie weitere Konsequenzen aus, werden sie zu Informationen: Informationen sind Daten mit Kontext.
    \end{block}
    \begin{block}{Wissen}
      Werden Informationen mit anderen Informationen verknüpft oder mit Angaben zu Quellen, Zuverlässigkeit, Entstehungsmethodik, \dots verbunden, entsteht aus Informationen Wissen: Wissen ist geprüfte Information.
    \end{block}
  \end{frame}


  \begin{frame}{Beispiel: Einwohnerdaten}
  \metroset{block=fill}
    \begin{block}{Standard-Datensätze}
      \begin{description}
        \item[Quelle:] Exportroutine im Melderegister
        \item[Abruf:] Zwei Optionen
          \begin{itemize}
            \item ``Dauerauftrag'' im vorgegebenen Turnus: Datenlieferung entsteht im Meldeamt, Transportweg ist festzulegen und zu sichern
            \item Eigener Abruf bei Bedarf: Zugang der Statistik zum EWO ist erforderlich, Exportdaten entstehen dafür schon im abgesicherten Bereich.
          \end{itemize}
         \item[Plausibilisierung:] Überprüfung/Korrektur von Schreibweisen, Zahlendrehern, evtl. Löschung unerwünschter Datensätze
         \item[Anreicherung:] Erzeugung zusätzlicher Informationen\footnote{Spätestens ab diesem Zeitpunkt unterliegen die Daten dem Rückspielverbot.}  wie Haushaltszugehörigkeit, Migrationshintergrund
        \end{description}
    \end{block}
  \end{frame}
  
    \begin{frame}{Informationelle Selbstbestimmung}
        \begin{block}{Informationsreduzierung}
            \begin{itemize}
                \item Pseudonymisierung durch Entfernen identifizierender Daten (insb. Namen)
                \item Entfernen von Hilfsmerkmalen
                \item Aggregation zu Makrodateien (Merkmalskombinationen statt Einzelfällen)
                \item Ganz allgemein: Abbau von Kontext
            \end{itemize}
        \end{block}
        \begin{block}{Informationsanreicherung}
            \begin{itemize}
                \item Deanonymisierung
                \item Verknüpfung voneinander unabhängiger Datenbestände über Schlüsselmerkmale, Georeferenz, \dots
                \``Drilldown''
                \item Ganz allgemein: Aufbau von Kontext
            \end{itemize}
        \end{block}
  \end{frame}

\section{Schlüsselsysteme}
    \begin{frame}{Beispiele}
        \begin{description}
            \item[WZ2008:] Klassifikation der Wirtschaftszweige
            \item[Nationalitätenschlüssel:] Staats- und Gebietssystematik
            \item[AGS:] Gemeindeverzeichnis-Informationssystem GV-ISys
        \end{description}
    \end{frame}
    \begin{frame}{WZ2008}
        \begin{itemize}
            \item Grundlage der Aggregation volkswirtschaflticher Gesamtrechnungen
            \item veröffentlicht von destatis (im Verbund mit einer Vielzahl internationaler Klassifikationssysteme \dots die aktuelle Veröffentlichung umfasst 828(!) Seiten)
            \item hierarchisch aufgebaut (z.B.) im Abschnitt P (Erziehung und Unterricht):
                \begin{description}
                    \item[85] Erziehung und Unterricht 
                    \item[85.5] Sonstiger Unterricht
                    \item[85.59] Sonstiger Unterricht a.n.g.\footnote{``anderweitig nicht genannt''}
                    \item[85.59.2] Berufliche Erwachsenenbildung
                \end{description}
        \end{itemize}
      
    \end{frame}
    \begin{frame}{Nationalitätenschlüssel}
        \begin{itemize}
            \item veröffentlicht von destatis (in internationaler Abstimmung)
            \item enthält:
            \begin{itemize}
                \item Staatsnamen in Kurzform (``Vatikanstadt''),
                \item \dots und in Vollform (``Staat Vatikanstadt''),
                \item Staatsangehörigkeit (``vatikanisch''),
                \item  Kontinent (``EUR''),
                \item zwei- und dreistelliger ISO-3166-1-Code (`VA''\footnote{Genutzt insb. als Top Level Domain (TLD) im WWW}, ``VAT''),
                \item Destatis-BEV-Code (``167'')\footnote{Dieser BEV-Code entspricht dem in der Bevölkerungsstatistik verwendeten Nationalitätenschlüssel im engeren Sinn, nicht zu verwechseln mit anderen Schlüsseln, die für den Internationalen Handel verwendet werden.} 
            \end{itemize}    
        \end{itemize}
    \end{frame}
    \begin{frame}{Aufbau des BEV-Code}
        \begin{description}
            \item[000:] Deutschland
            \item[1xx:] Europa (ohne D)
            \item[2xx:] Afrika
            \item[3xx:] Amerika(s)
            \item[4xx:] Asien
            \item[5xx:] Australien und Ozeanien
            \item[unterhalb der ``kontinentalen''-Ebene:] grobe alphabetische Reihenfolge
            \end{description}
    \end{frame}
    \begin{frame}{BEV-Code-Problem an Beispiel Jugoslawien}
        \begin{table}
    \centering
        \begin{tabular}{llll}
            \textbf{Staat}  & \textbf{Code} & \textbf{gültig von} & \textbf{gültig bis}\\
                Bosnien und Herzegowina & 122 & 01.03.1992 & \dots             \\
                Jugoslawien & 120 & \dots & 26.04.1992      \\
                Bundesrepublik Jugoslawien & 138 & 27.04.1992 & 04.02.2003      \\
                Kosovo & 150 & 17.02.2008 & \dots \\
                Kroatien & 130 & 25.06.1991 & \dots \\
                Montenegro & 141 & 03.06.2006 & \dots \\
                Nordmazedonien & 144 & 08.09.1991 & \dots \\
                Serbien und Montenegro & 132 & 05.02.2003 & 02.06.2006 \\ 
                Serbien (mit Kosovo) & 133 & 03.06.2006 & 16.02.2008 \\
                Serbien & 133 & 17.02.2008 & \dots \\
                Slowenien & 131 & 25.06.1991 & \dots \\
        \end{tabular}
    \end{table}
    \end{frame}
    \begin{frame}{GV-ISys}
        \begin{block}{Gemeindeverzeichnis-Informationssystem}
            \begin{itemize}
                \item veröffentlicht von destatis
                \item enthält:
                    \begin{itemize}
                        \item Amtlicher Regionalschlüssel (ARS)
                        \item Amtlicher Gemeindeschlüssel (AGS)
                        \item PLZ des Verwaltungssitzes
                        \item Fläche in km\textsuperscript{2}
                        \item Einwohnerzahl (insgesamt/männlich/weiblich)
                        \item siedlungsstrukturelle Typisierungen
                    \end{itemize}
            \end{itemize}
        \end{block}
    \end{frame}
    \begin{frame}[allowframebreaks]{Beispiele für den AGS}
        \begin{block}{Hansestadt Stralsund, AGS 13073088}
            \begin{description}
                \item[13] Mecklenburg-Vorpommern
                \item[0] Regierungsbezirk (gibt es nicht in MV, deshalb 0)
                \item[73] LKr. Vorpommern-Rügen (7 für LKr, davon die Nr. 3)
                \item[088] Gemeinde Nr. 88 im Kreis
            \end{description}
        \end{block}
        \framebreak
        \begin{block}{Paderborn, AGS 05774032}
            \begin{description}
                \item[05] Nordrhein-Westfalen
                \item[7] Regierungsbezirk Detmold
                \item[74] LKr. Paderborn (7 für LKr, davon die Nr. 4)
                \item[032] Gemeinde Nr. 32 im LKr\footnote{eigentlich die Nr.8, aber hier wird in 4er-Schritten durchgezählt.}
            \end{description}
        \end{block}        
        \framebreak
        \begin{block}{Neumünster, AGS 01004000}
            \begin{description}
                \item[01] Schleswig-Holstein
                \item[0] Regierungsbezirk (gibt es nicht in SH, deshalb 0)
                \item[04] 4. Kreisfreie Stadt (0 für KrfSt) in SH
                \item[000] kein weiterer Gemeindeschlüssel, da kreisfrei
            \end{description}
        \end{block}        
        \framebreak
        \begin{block}{Hamm, AGS 05915000}
            \begin{description}
                \item[05] Nordrhein-Westfalen
                \item[9] Regierungsbezirk Arnsberg
                \item[15] 5. Kreisfreie Stadt (1 für KrfSt) in NW
                \item[000] kein weiterer Gemeindeschlüssel, da kreisfrei
            \end{description}
        \end{block}
        \framebreak
        \begin{block}{Neuss, AGS 05162024}
            \begin{description}
                \item[05] Nordrhein-Westfalen
                \item[1] Regierungsbezirk Düsseldorf
                \item[62] LKr Rheinkreis Neuss
                \item[024] Gemeinde Nr. 24 im LKr
            \end{description}
        \end{block}
        \framebreak
        \begin{block}{Tübingen, AGS 08416041}
            \begin{description}
                \item[08] Baden-Württemberg
                \item[4] Regierungsbezirk Tübingen
                \item[16] LKr Tübingen
                \item[041] Gemeinde Nr. 41 im LKr
            \end{description}
        \end{block} 
        \framebreak
        \begin{block}{Erlangen, AGS 09562000}
            \begin{description}
                \item[09] Bayern
                \item[5] Regierungsbezirk Mittelfranken
                \item[62]  2. Kreisfreie Stadt (6 für KrfSt) in Mittelfranken
                \item[000] kein weiterer Gemeindeschlüssel, da kreisfrei
            \end{description}
        \end{block} 
        \framebreak
        \begin{block}{Schwabach, AGS 09565000}
            \begin{description}
                \item[09] Bayern
                \item[5] Regierungsbezirk Mittelfranken
                \item[62]  5. Kreisfreie Stadt (6 für KrfSt) in Mittelfranken
                \item[000] kein weiterer Gemeindeschlüssel, da kreisfrei
            \end{description}
        \end{block}
        \framebreak
        \begin{block}{Passau, AGS 09262000}
            \begin{description}
                \item[09] Bayern
                \item[2] Regierungsbezirk Niederbayern
                \item[62]  2. Kreisfreie Stadt (6 für KrfSt) in Niederbayern
                \item[000] kein weiterer Gemeindeschlüssel, da kreisfrei
            \end{description}
        \end{block}
        \framebreak
        \begin{block}{Salzgitter, AGS 03102000}
            \begin{description}
                \item[03] Niedersachsen
                \item[1] Statistische Region Braunschweig
                \item[02]  2. Kreisfreie Stadt (0 für KrfSt) in der Region
                \item[000] kein weiterer Gemeindeschlüssel, da kreisfrei
            \end{description}
        \end{block}
        \framebreak
        \begin{block}{Cottbus, AGS 12052000}
            \begin{description}
                \item[12] Brandenburg
                \item[0] Regierungsbezirk (gibt es nicht in BB, deshalb 0)
                \item[52]  2. Kreisfreie Stadt (5 für KrfSt) in Brandenburg
                \item[000] kein weiterer Gemeindeschlüssel, da kreisfrei
            \end{description}
        \end{block}
    \end{frame}
    
\include{KlGl}
\section{Softwareoptionen}
    \begin{frame}{Standardsoftware}
        \begin{itemize}
            \item Office-Pakete sind allgegenwärtig, für den Produktiveinsatz grundsätzlich auch geeignet. Limits bei der Anzahl der Zeilen gibt es eigentlich nicht mehr. Gestaltungsmöglichkeiten für Berichte, Tabellen und Diagramme sind durchaus brauchbar.
            \item An DB-Software (Oracle, MS-SQL, mySQL, \dots) führt eigentlich kein Weg vorbei.
        \end{itemize}
    \end{frame}
    \begin{frame}{Spezielle Statistiksoftware}
        \begin{itemize}
            \item SPSS (bzw. der OpenSource-Clon PSPP) sind in vielen Städten im Einsatz, jedoch in nicht unerheblichem Ausmaß kostenpflichtig
            \item R scheint sich in den letzten Jahren an den Hochschulen immer mehr durchzusetzen. Seit 2019 gibt es im VDSt die Kosis-Gemeinschaft ``Ko-R'', die zum Ziel hat, R als gemeinsame Entwicklungs- und Anwendungssprache zu etablieren.
            \item SAS
            \item Stata
            \item Statistica
        \end{itemize}
    \end{frame}
    \begin{frame}{Kosis-Software}
        \begin{description}
            \item[AGK\footnote{Das ``AG'' in ``AGK'' steht \textbf{nicht} für ``Andreas Gleich''. Dahingehende Gerüchte sind nur leicht übertrieben.}] \textbf{A}dressen, \textbf{G}ebäude, \textbf{K}leinräumige Gliederung
            \item[HHSTAT] Haushaltsstrukturen aus den Informationen der Melderegister
            \item[DUVA] Informationsmanagement
            \item[Kosis-APP] Kleinräumige Statistikdaten für unterwegs
            \item[KO-Umfrage] Organisation und Durchführung von Umfragen
            \item[KO-Wahl] Wahlanalyse
            \item[Sikurs] Kleinräumige Bevölkerungsprognose
            \item[KO.R] Analyse- und Auswertungstools mit R
        \end{description}
    \end{frame}
    \begin{frame}{HHSTAT}
        \includegraphics[width=150pt]{images/HHSTAT.png}
        \centering
        \begin{table}[]
            \centering
            \begin{tabular}{lp{9cm}}
                 \textbf{EwoPEaK:} & Protokollierte(!) Plausibilisierung, Korrektur, Ergänzung und Editierung von Einwohnerdateien (Bewegung und Bestand) \\
                 \textbf{Migrapro:} & Ableitung von Migrationshintergründen \\
                 \textbf{HHGEN:} &  Generierung von Haushalten aus dem Melderegister \\
                 \textbf{Gizeh:} & Konfigurierbare Pyramidendiagramme \\
            \end{tabular}
        \end{table}
    \end{frame}
    \begin{frame}{Duva}
        \includegraphics[width=150pt]{images/DUVA.png}
        \centering
        \begin{table}[]
            \centering
            \begin{tabular}{lp{9cm}}
                 \textbf{NWS:} & Metadatenbasiertes Nachweissystem, enthält Dateiobjekte, deren Satzbeschreibungen, Codierungen, Referenztabellen, \dots  \\
                 \textbf{FormGen:} & Formulargenerator für On- und Offline-Datenerfassung \\
                 \textbf{IERF:} & Datenerfassung am PC oder im Web \\
                 \textbf{ASW:} & Auswertungsassistent zur flexiblen Gestaltung von Tabellen und Diagrammen \\
                 \textbf{Info-Portal:} & Aufbau und Pflege eines Informationsangebot \\
            \end{tabular}
        \end{table}
    \end{frame}
    \begin{frame}{Kosis-APP}
        \includegraphics[width=150pt]{images/KosisApp.png}
        \centering
        \begin{itemize}
            \item Arbeitslose und Beschäftigte nach Geschlecht
            \item Einwohner nach Altersgruppen, Familienstand, Migrationshintergrund
            \item Geburten, Sterbefälle und Wanderungen
            \item Haushalte nach Haushaltsgröße
            \item Wohnungen nach Anzahl der Räume
            \item Zweitstimmenanteile bei der Bundestagswahl
        \end{itemize}
    \end{frame}
    \begin{frame}{KO-Umfrage}
        \includegraphics[width=150pt]{images/KOUmfrage.png}
        \centering
        \begin{itemize}
            \item Keine eigene Software-Entwicklung
            \item ``Softwareanwendergemeinschaft'', Schwerpunkt derzeit ``Blubbsoft''
        \end{itemize}
    \end{frame}
    \begin{frame}{KO-Wahl}
        \includegraphics[width=150pt]{images/KOWahl.png}
        \centering
        \begin{itemize}
            \item \textbf{Keine} eigene Software-Entwicklung
            \item IT-Fachverfahren zur Wahlorganisation und Ergebnisermittlung
            \item Berechnung von Wählerwanderungen
            \item Erhebungsinstrumente und Organisationshilfen für Wahltagsbefragungen
            \item Wandel im Verhältnis von Briefwahl und Lokalwahl
            \item Verfahren zur Hochburgenanalyse
            \item Umrechnung von Wahlergebnissen (Gebietsstandänderungen, Einrechnung von Briefwahlergebnissen in allgemeine Wahlbezirke
        \end{itemize}
    \end{frame}
    \begin{frame}{Sikurs}
        \includegraphics[width=150pt]{images/Sikurs.png}
        \centering
        \begin{itemize}
            \item Kleinräumige Bevölkerungsprognose
            \item Baukastenprinzip aus 17 Bausteinen
                \begin{itemize}
                    \item AUßenwanderung
                    \item Binnenwanderung
                    \item Neubaubezug
                    \item \dots
                \end{itemize}
            \item Unterstützung verschiedener Szenarien
            \item Sikurs-Modell wird auch von Landesämtern eingesetzt
        \end{itemize}
    \end{frame}
    \begin{frame}{KO.R}
        \includegraphics[width=100pt]{images/KoR.png}
        \centering
            \begin{itemize}
                \item Gemeinsame Entwicklung und Anwendung von Methoden zur Datenauswertung mit \includegraphics[height=\baselineskip]{images/Rlogo.png}
                \item Ausbau von Statistik-Packages in \includegraphics[height=\baselineskip]{images/Rlogo.png} für kommunalstatistische Anwendungen
                \item Gemeinsame Planung und Durchführung von Plausibilisierungen von Datenbeständen
                \item Automatisierungen im Reporting und die Gestaltung von Dashboards
                \item Aufbau und Pflege einer gemeinsamen Informations- und Austauschplattform
            \end{itemize}
    \end{frame}
\section[Datenquellen]{Datenquellen}
    \begin{frame}{Aus der Verwaltung}
    \metroset{block=fill}
        \begin{block}{Datenbanken von Verwaltungsverfahren}
            \begin{itemize}
                \item Meldewesen (Bestand und Bewegung)
                \item Kfz-Zulassung
                \item Gewerbeamt
                \item Wahlergebnisse
                \item \dots
            \end{itemize}
        \end{block}
    \end{frame}
    \begin{frame}{Technische Quellen (Beispiele)}
    \metroset{block=fill}
        \begin{block}{Eigene Quellen}
            \begin{itemize}
                \item Kommunale Verkehrsleitsysteme
                \item 
            \end{itemize}
        \end{block}
        \begin{block}{Zugekaufte Daten}
            \begin{itemize}
                \item Handydaten
                \item 
            \end{itemize}
        \end{block}
    \end{frame}
    \begin{frame}{Staatliche Statistik}
    \metroset{block=fill}
        \begin{block}{Recherchesysteme}
            \begin{itemize}
                \item Genesis Online
                \item Deep Links / Web Services
            \end{itemize}
        \end{block}
        \begin{block}{Einzeldatenbezug}
            \begin{itemize}
                \item An abgeschottete Statistikstellen grundsätzlich möglich, jedoch regelmäßig mit hohem Aufwand verbunden.
                \item Beispiel: Schülerdaten
            \end{itemize}
        \end{block}
    \end{frame}

% \include{12InterkommunalerDatenaustausch}
% \section[VDSt]{Verband Deutscher Städtestatistiker}
\begin{frame}{VDSt}
\includegraphics[width=120pt]{images/VDSt.png}
\newline
	Der Verband Deutscher Städtestatistiker\footnote{nicht zu verwechseln mit dem Verband Deutscher Sporttaucher (ebenfalls VDST)},  vertritt die Interessen der Kommunalstatistik in allen übergeordneten Gremien und Institutionen (kommunale Spitzenverbände, staatliche Statistik, Gesetzgebungsverfahren, \dots).
	
	\begin{table}[]
	    \centering
	    \begin{tabular}{lp{7cm}}
	         \textbf{Mitgliedschaft:}&  Persönliche Mitgliedschaft durch Beitrittsantrag, zwei Unterstützerunterschriften und Aufnahmebeschluss des Vorstands\\
	         \textbf{Vorteil:}& Zugang zu Mitgliederverzeichnis, Mitgliederbereich, Mitgliedszeitschrift\\
	    \end{tabular}
	\end{table}
	
\end{frame}

\begin{frame}{KOSIS}
\includegraphics[width=120pt]{images/KOSIS.png}
\centering
	\begin{table}[]
	    \centering
	    \begin{tabular}{lp{7cm}}
	         \textbf{Mitgliedschaft:}&  Persönliche Mitgliedschaft durch Beitrittsantrag, zwei Unterstützerunterschriften und Aufnahmebeschluss des Vorstands\\
	         \textbf{Vorteil:}& Zugang zu Mitgliederverzeichnis, Mitgliederbereich, Mitgliedszeitschrift\\
	    \end{tabular}
	\end{table}
\end{frame}
% \include{14AGs}
% \include{15ZensusVorbereitung}
% \include{16Erfahrungsaustausch}

% \section{Referenzen}
\subsection{Rechtliches}
\begin{frame}{Gesetzestexte}
    \begin{itemize}
        \item \href{https://ec.europa.eu/eurostat/de/web/products-statistical-books/-/KS-31-09-254}{EU-Statistikverordnung}
        \item \href{https://www.gesetze-im-internet.de/bstatg_1987/BStatG.pdf}{Bundesstatistikgesetz}
        \item \href{http://www.landesrecht-bw.de/jportal/portal/t/sx7/page/bsbawueprod.psml/screen/JWPDFScreen/filename/StatG_BW_jlr-StatGBWrahmen.pdf}{LStatG - (BaWü)}
        \item \href{https://www.gesetze-bayern.de/Content/Pdf/BayStatG?all=True}{BayStatG}
        \item \href{https://www.gesetze-im-internet.de/bdsg_2018/BDSG.pdf}{Bundesdatenschutzgesetz}
    \end{itemize}
\end{frame}

\begin{frame}[allowframebreaks]{Art. 46 BDSG - Begriffsbestimmungen}
\begin{enumerate}
    \item \alert{personenbezogene Daten} alle Informationen, die sich auf eine identifizierte oder identifizierbare natürliche Person (betroffene Person) beziehen; als identifizierbar wird eine natürliche Person angesehen, die direkt oder indirekt, insbesondere mittels Zuordnung zu einer Kennung wie einem Namen, zu einer Kennnummer, zu Standortdaten, zu einer Online-Kennung oder zu einem oder mehreren besonderen Merkmalen, die Ausdruck der physischen, physiologischen, genetischen, psychischen, wirtschaftlichen, kulturellen oder sozialen Identität dieser Person sind, identifiziert werden kann;
    \item \alert{Verarbeitung} jeden mit oder ohne Hilfe automatisierter Verfahren ausgeführten Vorgang oder jede solche Vorgangsreihe im Zusammenhang mit personenbezogenen Daten wie das Erheben, das Erfassen, die Organisation, das Ordnen, die Speicherung, die Anpassung, die Veränderung, das Auslesen, das Abfragen, die Verwendung, die Offenlegung durch Übermittlung, Verbreitung oder eine andere Form der Bereitstellung, den Abgleich, die Verknüpfung, die Einschränkung, das Löschen oder die Vernichtung;
    \item \alert{Einschränkung der Verarbeitung} die Markierung gespeicherter personenbezogener Daten mit dem Ziel, ihre künftige Verarbeitung einzuschränken;
    \item \alert{Profiling} jede Art der automatisierten Verarbeitung personenbezogener Daten, bei der diese Daten verwendet werden, um bestimmte persönliche Aspekte, die sich auf eine natürliche Person beziehen, zu bewerten, insbesondere um Aspekte der Arbeitsleistung, der wirtschaftlichen Lage, der Gesundheit, der persönlichen Vorlieben, der Interessen, der Zuverlässigkeit, des Verhaltens, der Aufenthaltsorte oder der Ortswechsel dieser natürlichen Person zu analysieren oder vorherzusagen;
    \item \alert{Pseudonymisierung} die Verarbeitung personenbezogener Daten in einer Weise, in der die Daten ohne Hinzuziehung zusätzlicher Informationen nicht mehr einer spezifischen betroffenen Person zugeordnet werden können, sofern diese zusätzlichen Informationen gesondert aufbewahrt werden und technischen und organisatorischen Maßnahmen unterliegen, die gewährleisten, dass die Daten keiner betroffenen Person zugewiesen werden können;
    \item \alert{Dateisystem} jede strukturierte Sammlung personenbezogener Daten, die nach bestimmten Kriterien zugänglich sind, unabhängig davon, ob diese Sammlung zentral, dezentral oder nach funktionalen oder geografischen Gesichtspunkten geordnet geführt wird;
    \item \alert{Verantwortlicher} die natürliche oder juristische Person, Behörde, Einrichtung oder andere Stelle, die allein oder gemeinsam mit anderen über die Zwecke und Mittel der Verarbeitung von personenbezogenen Daten entscheidet;
    \item \alert{Auftragsverarbeiter} eine natürliche oder juristische Person, Behörde, Einrichtung oder andere Stelle, die personenbezogene Daten im Auftrag des Verantwortlichen verarbeitet;
    \item \alert{Empfänger} eine natürliche oder juristische Person, Behörde, Einrichtung oder andere Stelle, der personenbezogene Daten offengelegt werden, unabhängig davon, ob es sich bei ihr um einen Dritten handelt oder nicht; Behörden, die im Rahmen eines bestimmten Untersuchungsauftrags nach dem Unionsrecht oder anderen Rechtsvorschriften personenbezogene Daten erhalten, gelten jedoch nicht als Empfänger; die Verarbeitung dieser Daten durch die genannten Behörden erfolgt im Einklang mit den geltenden Datenschutzvorschriften gemäß den Zwecken der Verarbeitung;
    \item \alert{Verletzung des Schutzes personenbezogener Daten} eine Verletzung der Sicherheit, die zur unbeabsichtigten oder unrechtmäßigen Vernichtung, zum Verlust, zur Veränderung oder zur unbefugten Offenlegung von oder zum unbefugten Zugang zu personenbezogenen Daten geführt hat, die verarbeitet wurden;
    \item \alert{genetische Daten} personenbezogene Daten zu den ererbten oder erworbenen genetischen Eigenschaften einer natürlichen Person, die eindeutige Informationen über die Physiologie oder die Gesundheit dieser Person liefern, insbesondere solche, die aus der Analyse einer biologischen Probe der Person gewonnen wurden;
    \item \alert{biometrische Daten} mit speziellen technischen Verfahren gewonnene personenbezogene Daten zu den physischen, physiologischen oder verhaltenstypischen Merkmalen einer natürlichen Person, die die eindeutige Identifizierung dieser natürlichen Person ermöglichen oder bestätigen, insbesondere Gesichtsbilder oder daktyloskopische Daten;
    \item \alert{Gesundheitsdaten} personenbezogene Daten, die sich auf die körperliche oder geistige Gesundheit einer natürlichen Person, einschließlich der Erbringung von Gesundheitsdienstleistungen, beziehen und aus denen Informationen über deren Gesundheitszustand hervorgehen;
    \item \alert{besondere Kategorien personenbezogener Daten}
        \begin{itemize}
            \item Daten, aus denen die rassische oder ethnische Herkunft, politische Meinungen, religiöse oder weltanschauliche Überzeugungen oder die Gewerkschaftszugehörigkeit hervorgehen,
            \item genetische Daten,
            \item biometrische Daten zur eindeutigen Identifizierung einer natürlichen Person,
            \item Gesundheitsdaten und
            \item Daten zum Sexualleben oder zur sexuellen Orientierung;
        \end{itemize}
    \item \alert{Aufsichtsbehörde} eine von einem Mitgliedstaat gemäß Artikel 41 der Richtlinie (EU) 2016/680 eingerichtete unabhängige staatliche Stelle;
    \item \alert{internationale Organisation} eine völkerrechtliche Organisation und ihre nachgeordneten Stellen sowie jede sonstige Einrichtung, die durch eine von zwei oder mehr Staaten geschlossene Übereinkunft oder auf der Grundlage einer solchen Übereinkunft geschaffen wurde;
    \item \alert{Einwilligung} jede freiwillig für den bestimmten Fall, in informierter Weise und unmissverständlich abgegebene Willensbekundung in Form einer Erklärung oder einer sonstigen eindeutigen bestätigenden Handlung, mit der die betroffene Person zu verstehen gibt, dass sie mit der Verarbeitung der sie betreffenden personenbezogenen Daten einverstanden ist.
\end{enumerate}

\end{frame}




% \appendix




% \section{Referenzen}
% \begin{frame}[allowframebreaks]{Links und Literatur}

% \printbibliography[
% heading=bibintoc,
% title={Link- und Literatursammlung}
% ]
% \end{frame}

\end{document}
