\documentclass[10pt]{beamer}

\usetheme[progressbar=frametitle]{metropolis}
\usepackage{appendixnumberbeamer}
\setbeamertemplate{frametitle continuation}{\usebeamerfont{frametitle}\insertcontinuationcount}

\usepackage{booktabs}
\usepackage[scale=2]{ccicons}

\usepackage{pgfplots}
\usepgfplotslibrary{dateplot}

\usepackage{xspace}
\newcommand{\themename}{\textbf{\textsc{metropolis}}\xspace}

\usepackage{graphicx}
\graphicspath{ {./images/} }




\usepackage{xpatch}

\makeatletter
\patchcmd\beamer@@tmpl@frametitle{\insertframetitle}{\insertsection: \insertframetitle}{}{}
\makeatother


\usepackage[backend=biber,style=alphabetic,sorting=ynt]{biblatex}
\addbibresource{Verweise.bib}


\title{Grundlagen der Kommunalstatistik }
\subtitle{Einsteigerseminar }
\date{\today}
\date{}
\author{Klaus Brückner}
\institute{Statistikstelle der Stadt Passau}
% \titlegraphic{\hfill\includegraphics[height=1.5cm]{logo.pdf}}

\setbeamertemplate{frame footer}{Grundlagen der Kommunalstatistk - Bamberg, Herbst 2020}
% \setbeamertemplate{frametitle continuation}{\arabic{\insertcontinuationcount}}

\begin{document}

\maketitle

\begin{frame}
  \setbeamertemplate{section in toc}[sections numbered]
  \tableofcontents[hideallsubsections]
\end{frame}

\section{Intro}

\begin{frame}[fragile]{Wer steht hier?}
    \begin{itemize}
        \item Klaus Brückner
        \begin{itemize}
            \item Leiter und gleichzeitig einziger Mitarbeiter der Statistikstelle der Stadt Passau
            \item \emph{kein} gelernter Statistiker oder Geograph
            \item Zensus-Erhebungsstellenleiter 2011 und wohl auch 2021/22
            \item nebenher zuständig für Breitbandausbau und öffentliches WLAN
        \end{itemize}
        \item Schwerpunkte der statistischen Arbeit
        \begin{itemize}
            \item Demografiebeobachtung (z.B. für Kita-Planung)
            \item Qualifizierter Mietspiegel
            \item Automatisierte Verkehrszählung
        \end{itemize}
        \item Zusatzaufgaben, "Nebenjobs"
        \begin{itemize}
            \item Breitbandausbau
            \item Öffentliches WLAN
        \end{itemize}
    \end{itemize}
\end{frame}

\begin{frame}{Ein Erklärungsversuch}
    \begin{exampleblock}{Aus https://el.wikipedia.org/wiki/Στατιστική:}
        ´´Ο όρος στατιστική είναι αρχαία ελληνική λέξη που ετυμολογείται από το αρχαίο ρήμα ίστημι και του εξ αυτού παραγώγου ρήματος στατίζω που σημαίνει τοποθετώ, ταξινομώ, συμπεραίνω.''
        \begin{description}
            \item[τοποθετώ:] auf den (richtigen) Platz stellen
            \item[ταξινομώ:] die (richtige) Ordnung benennen
            \item[συμπεραίνω:] die (richtige) Schlussfolgerung ziehen
        \end{description}
    \end{exampleblock}
\end{frame}

\begin{frame}{Auch richtige Zahlen können sinnlos sein}
    \begin{exampleblock}{Frage: Wieviele Päpste je Quadratkilometer leben im Vatikan?}
        Die Antwort ($\frac{1 Papst}{0,44 km^{2}}$ = 2,27) ist ebenso richtig wie sinnlos.
    \end{exampleblock}
     \begin{exampleblock}{Frage: Wie drückt man einen Notendurchschnitt von 2,25 auf einer Skala von 1-5 aus?}
        $\frac{2,25}{5}*6$ = 1,875
    \end{exampleblock}
     \begin{exampleblock}{Frage: ... und wie auf einer Skala von A-E?}
        ???
    \end{exampleblock}
    \begin{alertblock}{Alter chinesischer Fluch:}
        Mögest du in statistisch interessanten Zeiten leben!
    \end{alertblock}
    
\end{frame}
\section{Rechtsrahmen}
\begin{frame}{Inhalt}
    \tableofcontents[currentsection, hideothersubsections]
\end{frame}



\subsection{Rechtliche Ebenen}
\begin{frame}[fragile]{Rechtliche Ebenen}
    \metroset{block=fill}
    \begin{block}{Europarecht: EU-Statistikverordnung}
    (und Regelungen mit Statistikbezug in anderen Verordnungen, wie z.B. EU-DSGVO, Durchführungsverordnungen zum Zensus, ...)
        \begin{block}{Bundesrecht: Bundesstatistikgesetz}
        (und Regelungen mit Statistikbezug in anderen Bundesgesetzen, z.B. zum Zensus, SGB, ...)

            \begin{block}{Landesrecht: Statistikgesetze der Länder, z.B. BayStatG}
            (und Regelungen mit Statistikbezug in anderen Landesgesetzen, z.B. zum Zensus)
                \begin{block}{Satzungen und Verordnungen auf kommunaler Ebene}
                (z.B. Mietspiegel-Erhebungssatzung)
                \end{block}
            \end{block}
        \end{block}
    \end{block}
\end{frame}

\subsection{Die Statistikgesetze im Einzelnen}
\begin{frame}{EU-Statistikverordnung}
    \metroset{block=fill}
    Als ``Hausgesetz'' von Eurostat entwickelt sie keine unmittelbare Wirkung auf die Kommunalstatistik, enthält jedoch eine Reihe von gelungenen Grundsätzen und Begriffsbestimmungen
    % \begin{description}
    %     \item[Fachliche Unabhängigkeit:] Statistiken werden auf unabhängige Weise entwickelt, erstellt und verbreitet, insbesondere was die Wahl der Verfahren, Methoden und Quellen sowie den Zeitpunkt und Inhalt aller Verbreitungsformen angeht: Keine Einflussnahme durch Interessengruppen!
    %     \item[Unparteilichkeit:] Statistiken werden auf neutrale Weise entwickelt, erstellt und verbreitet: Alle Nutzer werden gleich behandelt.
    %     \item[Objektivität:] Statistiken werden in systematischer, zuverlässiger und unvoreingenommener Weise entwickelt, erstellt und verbreitet. Fachliche und ethische Standards werden eingehalten, Grundsätze und Verfahren sind für alle Beteiligten transparent.
    %     \item[Zuverlässigkeit:] Statistiken messen die Gegebenheiten, die sie abbilden sollen, so getreu, genau und konsistent wie möglich, wobei zur Wahl der Quellen, Methoden und Verfahren wissenschaftliche Kriterien herangezogen werden.
    % \end{description}

\end{frame}

\begin{frame}{Bundesstatistikgesetz}
\metroset{block=fill}
\begin{block}{Föderalismusprinzip}
``Durchregieren'' vom Bund zu den Kommunen ist im Föderalismusprinzip nicht vorgesehen. Kommunale Statistische Ämter arbeiten nicht unmittelbar für den Bund, deshalb ist das BStatG auch nicht von unmittelbarem Interesse.
\end{block}
\begin{block}{ABER: Rahmen für die Landesgesetze}
Das BStatG enthält eine Reihe von Grundsätzen, Begriffsbestimmungen, \dots, die ebenso in den verschiedenen Landesgesetzen stehen könnten (und zum Teil auch dort stehen).
\end{block}

\end{frame}

\begin{frame}{Landesstatistikgesetze}
    \metroset{block=fill}
    \begin{block}{Landes-/Bundes- vs. Kommunalstatistiken} Die Landesstatistikgesetze regeln insbesondere auch Art und Umfang der Tätigkeit kommunaler Stellen im staatlichen Auftrag (``übertragener Wirkungskreis'').
    \end{block}
    \begin{block}{Erhebungsstellen}
    EHst werden häufig zweimal beschrieben:
            \begin{itemize}
                \item Art. 20 BayStatG oder ThürStatG: Statistikstellen im staatlichen Auftrag
                \item Art. 24 BayStatG oder ThürStatG: Statistkstellen im Rahmen der kommunalen Selbstverwaltung
            \end{itemize}
    \end{block}
    \begin{block}{Einrichtung durch Satzung}
    EHSt-Einrichtung durch Satzung wird in den Landesstatistikgesetzen häufig vorgeschrieben.
    \end{block}

\end{frame}

\begin{frame}{Satzungen auf kommunaler Ebene}
    \metroset{block=fill}
    \begin{block}{Konstituierende Satzungen}
        z.B. aus Art. 24 Satz 2 Satz 1 BayStatG ergibt sich sowohl Verpflichtung als auch Ermächtigung für den Erlass der Statistiksatzung der Stadt Passau.
    \end{block}
    \begin{block}{Projektbezogene Satzungen, z.B.}
        \begin{itemize}
            \item Mietspiegelerhebungssatzung
            \item Bürgerbefragungssatzung
        \end{itemize}
    \end{block}
\end{frame}
\begin{frame}{Weitere Regelungen}
    \metroset{block=fill}
    \begin{block}{Dienstanweisungen, Betriebskonzepte}
        \begin{itemize}
            \item Als rein organisatorische Richtlinien eigentlich nicht dem Rechtsrahmen zuzurechnen
            \item Konkretisierung von Anforderungen z.B. aus Art. 89 EU-DSGVO (``\dots geeignete Garantien für Rechte und Freiheiten der betroffenen Person\dots'')
        \end{itemize}
    \end{block}
    

\end{frame}

\section{Datenschutz}
\begin{frame}[allowframebreaks]{EU-DSGVO}
Die EU-DSGVO nennt den Begriff der Statistik immerhin neunmal im eigentlichen Gesetztestext (ohne Begründung und Fußnoten)
\begin{enumerate}
    \item Art. 5 Abs. 1 lit. b: ``Weiterverarbeitung für \dots \alert{statistische} Zwecke gilt gemäß Artikel 89 Absatz 1 nicht als unvereinbar mit den ursprünglichen Zwecken („Zweckbindung“)''
    \item Art. 5 Abs. 1 lit. e: ``Personenbezogene Daten dürfen länger gespeichert werden, soweit die personenbezogenen Daten vorbehaltlich der Durchführung geeigneter technischer und organisatorischer Maßnahmen\footnote{sprich: Abschottung}, die von dieser Verordnung zum Schutz der Rechte und Freiheiten der betroffenen Person gefordert werden, ausschließlich \dots für \alert{statistische} Zwecke gemäß Artikel 89 Absatz 1 verarbeitet werden („Speicherbegrenzung“)''
    \item Art. 9 Abs.1 (Untersagung der Verarbeitung besonderer Daten); Abs. 2 setzt Abs. 1 in bestimmten Fällen außer Kraft, z.B. lit. j: ``Die Verarbeitung ist \dots (unter bestimmten Einschränkungen und Voraussetzungen) \dots für \alert{statistische} Zwecke gemäß Artikel 89 Absatz 1 erforderlich.''
    \item Art. 14 (Informationspflicht), Abs. 5 (Die Absätze 1 bis 4 finden keine Anwendung, wenn und soweit) \dots lit. b: ``die Erteilung dieser Informationen sich als unmöglich erweist oder einen unverhältnismäßigen Aufwand erfordern würde; dies gilt insbesondere für die Verarbeitung für im öffentlichen Interesse liegende Archivzwecke, für wissenschaftliche oder historische Forschungszwecke oder für \alert{statistische} Zwecke vorbehaltlich der in Artikel 89 Absatz 1 genannten Bedingungen und Garantien\footnote{sprich: Abschottung} \dots''
    \item Art. 17 (Recht auf Löschung "Vergessenwerden") Abs. 1 und 2 gilt laut Abs 3 nicht (lit. d) ``	
für im öffentlichen Interesse liegende Archivzwecke, wissenschaftliche oder historische Forschungszwecke oder für \alert{statistische} Zwecke gemäß Artikel 89 Absatz 1, soweit das in Absatz 1 genannte Recht voraussichtlich die Verwirklichung der Ziele dieser Verarbeitung unmöglich macht oder ernsthaft beeinträchtigt''
    \item Art. 21 Abs. 6 (Widerspruchsrecht): ``Die betroffene Person hat das Recht, aus Gründen, die sich aus ihrer besonderen Situation ergeben, gegen die sie betreffende Verarbeitung sie betreffender personenbezogener Daten, \dots \alert{statistischen} Zwecken gemäß Artikel 89 Absatz 1 erfolgt, Widerspruch einzulegen, es sei denn, die Verarbeitung ist zur Erfüllung einer im öffentlichen Interesse liegenden Aufgabe erforderlich.''
  \end{enumerate}
\end{frame}

\begin{frame}[allowframebreaks]{Art. 89 EU-DSGVO}
    Garantien und Ausnahmen in Bezug auf die Verarbeitung zu im öffentlichen Interesse liegenden Archivzwecken, zu wissenschaftlichen oder historischen Forschungszwecken und zu \alert{statistischen} Zwecken
    \begin{enumerate}
        \item Die Verarbeitung \dots zu \alert{statistischen} Zwecken unterliegt geeigneten Garantien für die Rechte und Freiheiten der betroffenen Person gemäß dieser Verordnung. Mit diesen Garantien wird sichergestellt, dass technische und organisatorische Maßnahmen bestehen, mit denen insbesondere die Achtung des Grundsatzes der Datenminimierung gewährleistet wird. \dots
        \item Werden personenbezogene Daten \dots zu \alert{statistischen} Zwecken verarbeitet, können vorbehaltlich der Bedingungen und Garantien gemäß Absatz 1 \dots Ausnahmen von den Rechten gemäß der Artikel 15\footnote{Auskunftsrecht}, 16\footnote{Recht auf Berichtigung}, 18\footnote{Recht auf Einschränkung der Bearbeitung} und 21\footnote{Widerspruchsrecht} vorgesehen werden, als diese Rechte voraussichtlich die Verwirklichung der spezifischen Zwecke unmöglich machen oder ernsthaft beeinträchtigen und solche Ausnahmen für die Erfüllung dieser Zwecke notwendig sind.
    \end{enumerate}
\end{frame}


\section[Abschottung]{Abschottung}
\begin{frame}[allowframebreaks]{In der EU-DSGVO?}
\begin{description}
    \item [Abschottung] taucht als Begriff in vielen gesetzlichen Regelungen aller vier Ebenen auf, eine konkrete Definition ist jedoch kaum zu finden.
    \item [Art. 89 Abs. 1 DSGVO] Die Verarbeitung \dots zu statistischen Zwecken unterliegt geeigneten \alert{Garantien} \dots. Mit diesen \alert{Garantien} wird sichergestellt, dass technische und organisatorische Maßnahmen bestehen, mit denen insbesondere die Achtung des Grundsatzes der Datenminimierung gewährleistet wird.
    \item[Art. 89 Abs. 2] Werden personenbezogene Daten \dots zu statistischen Zwecken verarbeitet, können \dots im Recht der Mitgliedstaaten insoweit Ausnahmen von den Rechten gemäß der Artikel 15, 16, 18 und 21 vorgesehen werden, als diese Rechte voraussichtlich die Verwirklichung der spezifischen Zwecke unmöglich machen oder ernsthaft beeinträchtigen und solche Ausnahmen für die Erfüllung dieser Zwecke notwendig sind.
    \item[Art. 15] \dots das Recht auf Auskunft über diese personenbezogenen Daten \dots
    \item[Art. 16] \dots das Recht, \dots die Berichtigung sie betreffender unrichtiger personenbezogener Daten \dots (und) \dots die Vervollständigung unvollständiger personenbezogener Daten \dots zu verlangen.
    \item[Art. 18]\dots das Recht, \dots die Einschränkung der Verarbeitung zu verlangen, \dots'
    \item[Art. 21]\dots das Recht, \dots gegen die Verarbeitung sie betreffender personenbezogener Daten, \dots Widerspruch einzulegen'
\end{description}
\end{frame}

\begin{frame}{Im BDSG}
    \begin{description}
        \item[\S 27 BDSG] greift Regelungen aus Art. 89 EU-DSGVO auf und konkretisiert sie.
        \item[\S 47] enthält eine Liste von Begriffsbestimmungen, von 1. 
    \end{description}
\end{frame}


\include{05Datenquellen}
\include{06RGS}
\include{07VDST}




% \appendix




\section{Referenzen}
\begin{frame}[allowframebreaks]{Links und Literatur}

\printbibliography[
heading=bibintoc,
title={Link- und Literatursammlung}
]

\end{frame}

\end{document}
