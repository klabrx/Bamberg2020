\section{Datenschutz durch Abschottung}
    \begin{frame}{EU}
    \metroset{block=fill}
        \begin{block}{Art. 89 Abs. 1 EU-DSGVO (``Garantenstellung''}
            ´´(\dots) Mit diesen Garantien wird sichergestellt, dass technische und organisatorische Maßnahmen bestehen, mit denen insbesondere die Achtung des Grundsatzes der Datenminimierung gewährleistet wird. Zu diesen Maßnahmen kann die Pseudonymisierung gehören, sofern es möglich ist, diese Zwecke auf diese Weise zu erfüllen. (\dots)''
        \end{block}
        \begin{block}{Bundesrecht, Landesrecht, Kommunales Recht}
            Vergleichbare Regelungen finden sich auf jeder Ebene.
        \end{block}
    
    \end{frame}
    \begin{frame}{Bund}
    \metroset{block=fill}
        \begin{block}{Abschottung}
            Der Begriff taucht kein einziges mal explizit auf.
        \end{block}
        \begin{block}{\S50 BDSG}
            Garantien, die ``\dots in einer räumlich und organisatorisch von den sonstigen Fachaufgaben getrennten Verarbeitung bestehen.''
        \end{block}
        \begin{block}{\S16 Abs. 5 Satz 2 BStatG: Übermittlung nur, wenn}
            ``\dots wenn durch Landesgesetz eine Trennung dieser Stellen von anderen kommunalen Verwaltungsstellen sichergestellt und das Statistikgeheimnis durch Organisation und Verfahren gewährleistet ist.''
        \end{block}    
    \end{frame}

    \begin{frame}{Länder}
    \metroset{block=fill}
        \begin{block}{Art. 20 Abs. 2 Satz 2 BayStatG}
            ``Statistikstellen\footnote{Anm.: im übertragenen Wirkungskreis, d.h. im staatlichen Auftrag} sind räumlich und organisatorisch von anderen Verwaltungsstellen zu trennen, gegen den Zutritt unbefugter Personen hinreichend zu sichern und mit Personal auszustatten, das die Gewähr für Zuverlässigkeit und Verschwiegenheit bietet.''
        \end{block}
        \begin{block}{Art. 24 Abs. 2 Satz 2 BayStatG}
            ``\textsuperscript{1}Statistikstellen\footnote{Anm.: im eigenen Wirkungskreis, d.h. im Rahmen der Selbstverwaltung} sind durch Satzung einzurichten, die auch die wesentlichen organisatorischen Bestimmungen, vornehmlich zur Wahrung des Statistikgeheimnisses zu treffen hat. \textsuperscript{1}Art. 20 Abs. 2 und 3 finden entsprechende Anwendung.''
        \end{block}    
    \end{frame}
    \begin{frame}{Kommunale Regelungen}
        \begin{block}{Satzungen, Dienstanweisungen, Betriebskonzepte}
            \begin{itemize}
                \item \dots ermöglichen das Eingehen auf örtliche Besonderheiten
                \\item \dots können relativ flexibel gestaltet und angepasst werden\footnote{Beispiel in Passau: Derzeit werden elektronische Schießzylinder eingeführt, die ein extrem flexibles Schließregime ermöglichen: Das kann in die Statistiksatzung aufgenommen werden.}
            \{itemize}
        \end{block}
    \end{frame}
