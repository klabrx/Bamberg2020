\section{Kleinräumige Gliederung}
\begin{frame}[allowframebreaks]{Kleinräumige Gliederung}
    \begin{exampleblock}{Alles, was passiert, passiert irgendwo.}
        Räumliche Gliederungssysteme erlauben die Zuordnung von Beobachtungen, Ereignissen, Erhebungsergebnissen zu geografischen Einheiten.
    \end{exampleblock}
    Drei Gliederungssysteme sind in der Kommunalstatistik relevant:
    \begin{description}
        \item[Nationalitätenschlüssel] steht für eine rechtliche Staatsangehörigkeit, ersatzweise auch für den Migrationshintergrund. Dieser 3-stellige sog. "BEV-Code" wird von statistischen Bundesamt (destatis) regelmäßig aktualisiert und veröffentlicht.
        \framebreak
        \item[Allgemeiner Gebietsschlüssel] bezeichnet die Zugehörigkeit zu einer Gemeinde. Der hierarchische Aufbau erlaubt auch die Zuordnung einer Gemeinde (am Beispiel Bayern) zum Kreis, Bezirk und Bundesland.
        \item[Kleinräumige Gliederung] steht für eine untergemeindliche Gebietseinteilung und Stadtteile, Bezirke, Distrikte, Reviere, Kieze \dots
    \end{description}
\end{frame}

\begin{frame}{Hierarchischer Aufbau}
    \begin{block}{Das Gebiet der Stadt Passau (AGS 09262000) besteht aus \dots}
        \begin{block}{5 Prognoseräumen (1-5),}
            diese bestehen aus \dots
            \begin{block}{16 Stadtteilen (11-52),}
                diese bestehen aus \dots
                    \begin{block}{Blöcken (11001-52035),}
                        diese bestehen aus \dots
                    \begin{block}{Blockseiten (1100101-5203507),}
                        an denen schießlich die Adressen als georeferenzierte Punkte liegen.
                    \end{block}
                    \end{block}
                \end{block}
        \end{block}
    \end{block}
    Somit steht die KlGl-Nummer 1100101 (bzw. 1-1-001-01) für Prognoseraum 1, Stadtteil 1, Block 001, Blockseite 01). Die Bezeichnungen in anderen Städten können abweichen, das Prinzip bleibt gleich.

    
\end{frame}

\begin{frame}[allowframebreaks]{Ausschnittsbeispiel}
\includegraphics[width=\textwidth]{images/klgl01.png}
\framebreak
\includegraphics[width=\textwidth]{images/klgl02.png}  
\end{frame}

\begin{frame}{Zuordnung von Gebietseinteilungen}
    \begin{exampleblock}{Grundsatz 1}
        Nicht die Adresse gehört unmittelbar zum Gebiet (Schulsprengel, Stimmbezirk, Verkehrszelle, \dots), sondern der Block\footnote{In begründeten Einzelfällen kann es auch die Blockseite sein, im Extremfall sogar ein Blockseitenabschnitt.}.
    \end{exampleblock}
    \begin{exampleblock}{Grundsatz 2}
        Jedes Gebiet setzt sich aus ganzen, nicht mehr weiter unterteilten Blöcken zusammen. Kein Block gehört zu zwei verschiedenen Schulsprengeln.
    \end{exampleblock}
    \begin{exampleblock}{Grundsatz 3}
        Wenn alle Blöcke eines Stadtteils zum selben Gebiet gehören, erfolgt die Zuweisung nicht auf Block- sondern gleich auf Stadtteilebene und wir nach unten weitervererbt.
    \end{exampleblock}
\end{frame}