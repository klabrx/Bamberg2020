\section[Rechtliche Rahmenbedingungen]{Rechtliche Rahmenbedingungen}
\subsection{Rechtliche Ebenen}
\begin{frame}[fragile]{Rechtliche Ebenen}
    \metroset{block=fill}
    \begin{block}{Europarecht: EU-Statistikverordnung}
    (und Regelungen mit Statistikbezug in anderen Verordnungen, wie z.B. EU-DSGVO\cite{eudsgvo}, Durchführungsverordnungen zum Zensus, ...)
        \begin{block}{Bundesrecht: Bundesstatistikgesetz}
        (und Regelungen mit Statistikbezug in anderen Bundesgesetzen, z.B. zum Zensus, SGB, ...)
        
            \begin{block}{Landesrecht: Statistikgesetze der Länder, z.B. BayStatG}
            (und Regelungen mit Statistikbezug in anderen Landesgesetzen, z.B. zum Zensus)
                \begin{block}{Satzungen und Verordnungen auf kommunaler Ebene}
                (z.B. Mietspiegel-Erhebungssatzung)
                \end{block}
            \end{block}
        \end{block}            
    \end{block}
\end{frame}

\subsection{Datenschutz}


\begin{frame}[allowframebreaks]{Datenschutz}

	\begin{block}{EU-DSGVO}
		Informationelle Selbstbestimmung, d.h. keine Verarbeitung personenbezogener/-beziehbarer Daten ohne Einwilligung oder berechtigte gesetzliche Regelung
	\end{block}
      \begin{exampleblock}{Verarbeitung}
        Erheben, Speichern, Auswerten, Zusammenfassen, Querverweisen, Aggregieren, Löschen, Verändern, ...
      \end{exampleblock}
      \begin{exampleblock}{Personendaten}
        Nur natürliche Personen, keine juristische Personen, Firmen, Betriebe, ...
      \end{exampleblock}      
      \begin{exampleblock}{Einwilligung}
        \begin{description}
            \item [Informiert:] (Was beinhaltet die Einwilligung? Wie komme ich aus der Nummer wieder raus?)
            \item [Dokumentiert:] (Nachweis der Einwilligung muss möglich sein)
        \end{description}
      \end{exampleblock}
      \begin{exampleblock}{berechtigte gesetzliche Regelung}
        \begin{description}
            \item [berechtigt:] i.d.R unproblematisch, im Zweifelsfall entsteht aus dem Selbstverwaltungrecht der Gemeinden das berechtigte Interesse an der Erhebung von Daten. REINE NEUGIERDE REICHT NICHT!
            \item [gesetzlich:] Die Regelung muss auf einem vorgeschriebenen Weg der Rechtsetzung zustandekommen (z.B. Beschluss des Stadtrats samt dazu passender Bekanntmachung)
            \item [Regelung:] Genaue Beschreibung dessen, was durch die gewünschte Datenverarbeitung erreicht werden soll
        \end{description}
      \end{exampleblock}      
\end{frame}