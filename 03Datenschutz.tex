\section{Datenschutz}
\begin{frame}[allowframebreaks]{EU-DSGVO}
Die EU-DSGVO nennt den Begriff der Statistik immerhin neunmal im eigentlichen Gesetztestext (ohne Begründung und Fußnoten)
\begin{enumerate}
    \item Art. 5 Abs. 1 lit. b: ``Weiterverarbeitung für \dots \alert{statistische} Zwecke gilt gemäß Artikel 89 Absatz 1 nicht als unvereinbar mit den ursprünglichen Zwecken („Zweckbindung“)''
    \item Art. 5 Abs. 1 lit. e: ``Personenbezogene Daten dürfen länger gespeichert werden, soweit die personenbezogenen Daten vorbehaltlich der Durchführung geeigneter technischer und organisatorischer Maßnahmen\footnote{sprich: Abschottung}, die von dieser Verordnung zum Schutz der Rechte und Freiheiten der betroffenen Person gefordert werden, ausschließlich \dots für \alert{statistische} Zwecke gemäß Artikel 89 Absatz 1 verarbeitet werden („Speicherbegrenzung“)''
    \item Art. 9 Abs.1 (Untersagung der Verarbeitung besonderer Daten); Abs. 2 setzt Abs. 1 in bestimmten Fällen außer Kraft, z.B. lit. j: ``Die Verarbeitung ist \dots (unter bestimmten Einschränkungen und Voraussetzungen) \dots für \alert{statistische} Zwecke gemäß Artikel 89 Absatz 1 erforderlich.''
    \item Art. 14 (Informationspflicht), Abs. 5 (Die Absätze 1 bis 4 finden keine Anwendung, wenn und soweit) \dots lit. b: ``die Erteilung dieser Informationen sich als unmöglich erweist oder einen unverhältnismäßigen Aufwand erfordern würde; dies gilt insbesondere für die Verarbeitung für im öffentlichen Interesse liegende Archivzwecke, für wissenschaftliche oder historische Forschungszwecke oder für \alert{statistische} Zwecke vorbehaltlich der in Artikel 89 Absatz 1 genannten Bedingungen und Garantien\footnote{sprich: Abschottung} \dots''
    \item Art. 17 (Recht auf Löschung "Vergessenwerden") Abs. 1 und 2 gilt laut Abs 3 nicht (lit. d) ``	
für im öffentlichen Interesse liegende Archivzwecke, wissenschaftliche oder historische Forschungszwecke oder für \alert{statistische} Zwecke gemäß Artikel 89 Absatz 1, soweit das in Absatz 1 genannte Recht voraussichtlich die Verwirklichung der Ziele dieser Verarbeitung unmöglich macht oder ernsthaft beeinträchtigt''
    \item Art. 21 Abs. 6 (Widerspruchsrecht): ``Die betroffene Person hat das Recht, aus Gründen, die sich aus ihrer besonderen Situation ergeben, gegen die sie betreffende Verarbeitung sie betreffender personenbezogener Daten, \dots \alert{statistischen} Zwecken gemäß Artikel 89 Absatz 1 erfolgt, Widerspruch einzulegen, es sei denn, die Verarbeitung ist zur Erfüllung einer im öffentlichen Interesse liegenden Aufgabe erforderlich.''
  \end{enumerate}
\end{frame}

\begin{frame}[allowframebreaks]{Art. 89 EU-DSGVO}
    Garantien und Ausnahmen in Bezug auf die Verarbeitung zu im öffentlichen Interesse liegenden Archivzwecken, zu wissenschaftlichen oder historischen Forschungszwecken und zu \alert{statistischen} Zwecken
    \begin{enumerate}
        \item Die Verarbeitung \dots zu \alert{statistischen} Zwecken unterliegt geeigneten Garantien für die Rechte und Freiheiten der betroffenen Person gemäß dieser Verordnung. Mit diesen Garantien wird sichergestellt, dass technische und organisatorische Maßnahmen bestehen, mit denen insbesondere die Achtung des Grundsatzes der Datenminimierung gewährleistet wird. \dots
        \item Werden personenbezogene Daten \dots zu \alert{statistischen} Zwecken verarbeitet, können vorbehaltlich der Bedingungen und Garantien gemäß Absatz 1 \dots Ausnahmen von den Rechten gemäß der Artikel 15\footnote{Auskunftsrecht}, 16\footnote{Recht auf Berichtigung}, 18\footnote{Recht auf Einschränkung der Bearbeitung} und 21\footnote{Widerspruchsrecht} vorgesehen werden, als diese Rechte voraussichtlich die Verwirklichung der spezifischen Zwecke unmöglich machen oder ernsthaft beeinträchtigen und solche Ausnahmen für die Erfüllung dieser Zwecke notwendig sind.
    \end{enumerate}
\end{frame}

