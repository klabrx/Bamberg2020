\section{Rechtsrahmen}
\subsection{Rechtliche Ebenen}
\begin{frame}[fragile]{Rechtliche Ebenen}
    \metroset{block=fill}
    \begin{block}{Europarecht: EU-Statistikverordnung}
    (und Regelungen mit Statistikbezug in anderen Verordnungen, wie z.B. EU-DSGVO, Durchführungsverordnungen zum Zensus, ...)
        \begin{block}{Bundesrecht: Bundesstatistikgesetz}
        (und Regelungen mit Statistikbezug in anderen Bundesgesetzen, z.B. zum Zensus, SGB, ...)
        
            \begin{block}{Landesrecht: Statistikgesetze der Länder, z.B. BayStatG}
            (und Regelungen mit Statistikbezug in anderen Landesgesetzen, z.B. zum Zensus)
                \begin{block}{Satzungen und Verordnungen auf kommunaler Ebene}
                (z.B. Mietspiegel-Erhebungssatzung)
                \end{block}
            \end{block}
        \end{block}            
    \end{block}
\end{frame}

\begin{frame}[allowframebreaks]{EU-Statistikverordnung}
Als "Hausgesetz" von Eurostat entwickelt sie keine unmittelbare Wirkung auf die Kommunalstatistik, enthält jedoch eine Reihe von gelungenen Grundsätzen und Begriffsbestimmungen:
    \begin{description}
        \item[Fachliche Unabhängigkeit:] Statistiken werden auf unabhängige Weise entwickelt, erstellt und verbreitet, insbesondere was die Wahl der Verfahren, Methoden und Quellen sowie den Zeitpunkt und Inhalt aller Verbreitungsformen angeht: Keine Einflussnahme durch Interessengruppen!
        \item[Unparteilichkeit:] Statistiken werden auf neutrale Weise entwickelt, erstellt und verbreitet: Alle Nutzer werden gleich behandelt.
        \item[Objektivität:] Statistiken werden in systematischer, zuverlässiger und unvoreingenommener Weise entwickelt, erstellt und verbreitet. Fachliche und ethische Standards werden eingehalten, Grundsätze und Verfahren sind für alle Beteiligten transparent.
        \item[Zuverlässigkeit:] Statistiken messen die Gegebenheiten, die sie abbilden sollen, so getreu, genau und konsistent wie möglich, wobei zur Wahl der Quellen, Methoden und Verfahren wissenschaftliche Kriterien herangezogen werden.
    \end{description}
    
\end{frame}

\begin{frame}[allowframebreaks]{Bundesstatistikgesetz}
Kommunale Statistische Ämter arbeiten nicht für den Bund, deshalb ist das BStatG nur am Rande von Interesse. Der \S1 beschreibt jedoch das Credo der öffentlichen Statistik perfekt:
    \begin{itemize}
        \item \S1 Satz 1:``Die \dots (Bundesstatistik) hat  \dots die Aufgabe, laufend Daten über Massenerscheinungen zu erheben, zu sammeln, aufzubereiten, darzustellen und zu analysieren. 
        \item \S1 Satz 2: ``Für sie gelten die Grundsätze der Neutralität, Objektivität und fachlichen Unabhängigkeit.''
        \item \S1 Satz 3: ``Sie gewinnt die Daten unter Verwendung wissenschaftlicher Erkenntnisse und unter Einsatz der jeweils sachgerechten Methoden und Informationstechniken.''
        \item \S1 Satz 4: Durch die Ergebnisse der Bundesstatistik werden gesellschaftliche, wirtschaftliche und ökologische Zusammenhänge für Bund, Länder einschließlich Gemeinden und Gemeindeverbände, Gesellschaft, Wirtschaft, Wissenschaft und Forschung aufgeschlüsselt.
        \item \S1 Satz 6:Die \dots erhobenen Einzelangaben dienen ausschließlich den durch dieses Gesetz oder eine andere eine Bundesstatistik anordnende Rechtsvorschrift festgelegten Zwecken.''
    \end{itemize}
\end{frame}

\begin{frame}{Landesstatistikgesetze}
    \begin{description}
        \item[Landes-/Bundes- vs. Kommunalstatistiken:] Die Landesstatistikgesetze regeln insbesondere auch Art und Umfang der Tätigkeit kommunaler Stellen im staatlichen Auftrag (``übertragener Wirkungskreis'').
        \item[Erhebungsstellen] werden häufig zweimal beschrieben:
            \begin{itemize}
                \item Art. 20 BayStatG oder ThürStatG: Statistikstellen im staatlichen Auftrag
                \item Art. 24 BayStatG oder ThürStatG: Statistkstellen im Rahmen der kommunalen Selbstverwaltung
            \end{itemize}
        \item[Einrichtung durch Satzung] wird in den Landesstatistikgesetzen häufig vorgeschrieben.
    \end{description}
    
\end{frame}

\begin{frame}[allowframebreaks]{Satzungen auf kommunaler Ebene}
    \begin{exampleblock}{Konstituierende Satzungen}
        z.B. Art. 24 Satz 2 Satz 1 BayStatG (```Statistikstellen sind durch Satzung einzurichten, die auch die wesentlichen organisatorischen Bestimmungen, vornehmlich zur Wahrung des Statistikgeheimnisses zu treffen hat.''') ergibt sich sowohl Verpflichtung als auch Ermächtigung für den Erlass der Statistiksatzung der Stadt Passau.
    \end{exampleblock}
    \begin{exampleblock}{Projektbezogene Satzungen, z.B.}
        \begin{itemize}
            \item Mietspiegelerhebungssatzung
            \item Bürgerbefragungssatzung
        \end{itemize}
    \end{exampleblock}
    \framebreak
    \begin{exampleblock}{Dienstanweisungen, Betriebskonzepte}
        \begin{itemize}
            \item Als rein organisatorische Richtlinien eigentlich nicht dem Rechtsrahmen zuzurechnen
            \item Konkretisierung von Anforderungen z.B. aus Art. 89 EU-DSGVO (``\dots geeignete Garantien für Rechte und Freiheiten der betroffenen Person\dots'')
        \end{itemize}
    \end{exampleblock}
    

\end{frame}