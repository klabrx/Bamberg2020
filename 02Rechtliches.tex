\section{Rechtsrahmen}
\begin{frame}{Inhalt}
    \tableofcontents[currentsection, hideothersubsections]
\end{frame}



\subsection{Rechtliche Ebenen}
\begin{frame}[fragile]{Rechtliche Ebenen}
    \metroset{block=fill}
    \begin{block}{Europarecht: EU-Statistikverordnung}
    (und Regelungen mit Statistikbezug in anderen Verordnungen, wie z.B. EU-DSGVO, Durchführungsverordnungen zum Zensus, ...)
        \begin{block}{Bundesrecht: Bundesstatistikgesetz}
        (und Regelungen mit Statistikbezug in anderen Bundesgesetzen, z.B. zum Zensus, SGB, ...)

            \begin{block}{Landesrecht: Statistikgesetze der Länder, z.B. BayStatG}
            (und Regelungen mit Statistikbezug in anderen Landesgesetzen, z.B. zum Zensus)
                \begin{block}{Satzungen und Verordnungen auf kommunaler Ebene}
                (z.B. Mietspiegel-Erhebungssatzung)
                \end{block}
            \end{block}
        \end{block}
    \end{block}
\end{frame}

\subsection{Die Statistikgesetze im Einzelnen}
\begin{frame}{EU-Statistikverordnung}
    \metroset{block=fill}
    Als ``Hausgesetz'' von Eurostat entwickelt sie keine unmittelbare Wirkung auf die Kommunalstatistik, enthält jedoch eine Reihe von gelungenen Grundsätzen und Begriffsbestimmungen
    % \begin{description}
    %     \item[Fachliche Unabhängigkeit:] Statistiken werden auf unabhängige Weise entwickelt, erstellt und verbreitet, insbesondere was die Wahl der Verfahren, Methoden und Quellen sowie den Zeitpunkt und Inhalt aller Verbreitungsformen angeht: Keine Einflussnahme durch Interessengruppen!
    %     \item[Unparteilichkeit:] Statistiken werden auf neutrale Weise entwickelt, erstellt und verbreitet: Alle Nutzer werden gleich behandelt.
    %     \item[Objektivität:] Statistiken werden in systematischer, zuverlässiger und unvoreingenommener Weise entwickelt, erstellt und verbreitet. Fachliche und ethische Standards werden eingehalten, Grundsätze und Verfahren sind für alle Beteiligten transparent.
    %     \item[Zuverlässigkeit:] Statistiken messen die Gegebenheiten, die sie abbilden sollen, so getreu, genau und konsistent wie möglich, wobei zur Wahl der Quellen, Methoden und Verfahren wissenschaftliche Kriterien herangezogen werden.
    % \end{description}

\end{frame}

\begin{frame}{Bundesstatistikgesetz}
\metroset{block=fill}
\begin{block}{Föderalismusprinzip}
``Durchregieren'' vom Bund zu den Kommunen ist im Föderalismusprinzip nicht vorgesehen. Kommunale Statistische Ämter arbeiten nicht unmittelbar für den Bund, deshalb ist das BStatG auch nicht von unmittelbarem Interesse.
\end{block}
\begin{block}{ABER: Rahmen für die Landesgesetze}
Das BStatG enthält eine Reihe von Grundsätzen, Begriffsbestimmungen, \dots, die ebenso in den verschiedenen Landesgesetzen stehen könnten (und zum Teil auch dort stehen).
\end{block}

\end{frame}

\begin{frame}{Landesstatistikgesetze}
    \metroset{block=fill}
    \begin{block}{Landes-/Bundes- vs. Kommunalstatistiken} Die Landesstatistikgesetze regeln insbesondere auch Art und Umfang der Tätigkeit kommunaler Stellen im staatlichen Auftrag (``übertragener Wirkungskreis'').
    \end{block}
    \begin{block}{Erhebungsstellen}
    EHst werden häufig zweimal beschrieben:
            \begin{itemize}
                \item Art. 20 BayStatG oder ThürStatG: Statistikstellen im staatlichen Auftrag
                \item Art. 24 BayStatG oder ThürStatG: Statistkstellen im Rahmen der kommunalen Selbstverwaltung
            \end{itemize}
    \end{block}
    \begin{block}{Einrichtung durch Satzung}
    EHSt-Einrichtung durch Satzung wird in den Landesstatistikgesetzen häufig vorgeschrieben.
    \end{block}

\end{frame}

\begin{frame}{Satzungen auf kommunaler Ebene}
    \metroset{block=fill}
    \begin{block}{Konstituierende Satzungen}
        z.B. aus Art. 24 Satz 2 Satz 1 BayStatG ergibt sich sowohl Verpflichtung als auch Ermächtigung für den Erlass der Statistiksatzung der Stadt Passau.
    \end{block}
    \begin{block}{Projektbezogene Satzungen, z.B.}
        \begin{itemize}
            \item Mietspiegelerhebungssatzung
            \item Bürgerbefragungssatzung
        \end{itemize}
    \end{block}
\end{frame}
\begin{frame}{Weitere Regelungen}
    \metroset{block=fill}
    \begin{block}{Dienstanweisungen, Betriebskonzepte}
        \begin{itemize}
            \item Als rein organisatorische Richtlinien eigentlich nicht dem Rechtsrahmen zuzurechnen
            \item Konkretisierung von Anforderungen z.B. aus Art. 89 EU-DSGVO (``\dots geeignete Garantien für Rechte und Freiheiten der betroffenen Person\dots'')
        \end{itemize}
    \end{block}
    

\end{frame}
