\section{Intro}

\begin{frame}[fragile]{Wer steht hier?}
    \begin{itemize}
        \item Klaus Brückner
        \begin{itemize}
            \item Leiter und gleichzeitig einziger Mitarbeiter der Statistikstelle der Stadt Passau
            \item \emph{kein} gelernter Statistiker oder Geograph
            \item Zensus-Erhebungsstellenleiter 2011 und wohl auch 2021/22
            \item Zusatzaufgaben, "Nebenjobs"
                \begin{itemize}
                    \item Breitbandausbau
                    \item Öffentliches WLAN
                \end{itemize}
        \end{itemize}
        \item Schwerpunkte der statistischen Arbeit
        \begin{itemize}
            \item Demografiebeobachtung (z.B. für Kita-Planung)
            \item Qualifizierter Mietspiegel
            \item Automatisierte Verkehrszählung
        \end{itemize}
    \end{itemize}
\end{frame}

\begin{frame}{Ein Erklärungsversuch}
    \begin{block}{aus https://de.wikipedia.org/wiki/Statistik}
        ``Das Wort Statistik stammt von lateinisch statisticum „den Staat betreffend“ und italienisch statista Staatsmann oder Politiker, was wiederum aus dem griechischen στατίζω (einordnen) kommt. Die deutsche Statistik, eingeführt von Gottfried Achenwall 1749, bezeichnete ursprünglich die „Lehre von den Daten über den Staat“. Im 19. Jahrhundert hatte der Schotte John Sinclair das Wort erstmals in seiner heutigen Bedeutung des allgemeinen Sammelns und Auswertens von Daten benutzt.''
    \end{block}
\end{frame} 

\begin{frame}{Ein Erklärungsversuch}
\metroset{block=fill}
    \begin{block}{Aus https://el.wikipedia.org/wiki/Στατιστική:}
        ´´Ο όρος στατιστική είναι αρχαία ελληνική λέξη που ετυμολογείται από το αρχαίο ρήμα ίστημι και του εξ αυτού παραγώγου ρήματος στατίζω που σημαίνει τοποθετώ, ταξινομώ, συμπεραίνω.''
        \begin{description}
            \item[τοποθετώ:] auf den (richtigen) Platz stellen
            \item[ταξινομώ:] die (richtige) Ordnung benennen
            \item[συμπεραίνω:] die (richtige) Schlussfolgerung ziehen
        \end{description}
    \end{block}
\end{frame}

\begin{frame}[allowframebreaks]{Seit wann gibt es Statistik?}
    \begin{itemize}
        \item Seit dem 19. Jahrhundert (John Sinclair)? Oder doch schon \dots
        \item Seit 1749 (Gottfied Achenwalls ``Lehre von den Daten über den Staat''), oder etwa \dots 
        \item seit 2000 Jahren: (Lukas 2, 1-3) ``Es geschah aber in jenen Tagen, dass Kaiser Augustus den Befehl erließ, den ganzen Erdkreis in Steuerlisten einzutragen. Diese Aufzeichnung war die erste; damals war Quirinius Statthalter von Syrien. Da ging jeder in seine Stadt, um sich eintragen zu lassen.''
        \item oder \dots
        \framebreak
        \item seit 2500 Jahren: (Daniel 1, 11-15) ``Da sagte Daniel zu dem Aufseher, den der Oberkämmerer über Daniel, Hananja, Mischaël und Asarja eingesetzt hatte: Versuch es doch einmal zehn Tage lang mit deinen Knechten: Man gebe uns Gemüse zu essen und Wasser zu trinken! Dann vergleiche unser Aussehen mit dem der Knaben, die von den Speisen des Königs essen! Je nachdem, was du dann siehst, verfahr weiter mit deinen Knechten! Der Aufseher nahm ihren Vorschlag an und versuchte es zehn Tage lang mit ihnen. Am Ende der zehn Tage sahen sie besser und wohlgenährter aus als all die Knaben, die von den Speisen des Königs aßen. ''
    \end{itemize}
    
\end{frame}

    \begin{frame}{Auch richtige Zahlen können sinnlos sein}
    \metroset{block=fill}
        \begin{block}{Frage: Wieviele Päpste je Quadratkilometer leben im Vatikan?}
            Die Antwort ($\frac{1 Papst}{0,44 km^{2}}$ = 2,27) ist ebenso richtig wie sinnlos.
        \end{block}
        \begin{block}{Frage: Wie drückt man einen Notendurchschnitt von 2,25 auf einer Skala von 1-5 aus?}
            $\frac{2,25}{5}*6$ = 1,875
        \end{block}
        \begin{block}{Frage: ... und wie auf einer Skala von A-E?}
            ???
        \end{block}
    \end{frame}

    \begin{frame}{Das informelle Informationsgesetz}
        \begin{table}[]
            \centering
            \begin{tabular}{rcl}
                Information - Emotion & = & \textbf{Fakten} \\
                Information + Erfahrung & = & \textbf{Meinung} \\
                Meinung - Information & = & \textbf{Ignoranz} \\
                Meinung $\neq$ Fakten & = & \textbf{Dummheit} \\
            \end{tabular}
        \end{table}
    \end{frame}
   
