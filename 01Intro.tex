\section{Intro}

\begin{frame}[fragile]{Wer steht hier?}
    \begin{itemize}
        \item Klaus Brückner
        \begin{itemize}
            \item Leiter und gleichzeitig einziger Mitarbeiter der Statistikstelle der Stadt Passau
            \item \emph{kein} gelernter Statistiker oder Geograph
            \item Zensus-Erhebungsstellenleiter 2011 und wohl auch 2021/22
            \item nebenher zuständig für Breitbandausbau und öffentliches WLAN
        \end{itemize}
        \item Schwerpunkte der statistischen Arbeit
        \begin{itemize}
            \item Demografiebeobachtung (z.B. für Kita-Planung)
            \item Qualifizierter Mietspiegel
            \item Automatisierte Verkehrszählung
        \end{itemize}
        \item Zusatzaufgaben, "Nebenjobs"
        \begin{itemize}
            \item Breitbandausbau
            \item Öffentliches WLAN
        \end{itemize}
    \end{itemize}
\end{frame}

\begin{frame}{Ein Erklärungsversuch}
    \begin{exampleblock}{Aus https://el.wikipedia.org/wiki/Στατιστική:}
        ´´Ο όρος στατιστική είναι αρχαία ελληνική λέξη που ετυμολογείται από το αρχαίο ρήμα ίστημι και του εξ αυτού παραγώγου ρήματος στατίζω που σημαίνει τοποθετώ, ταξινομώ, συμπεραίνω.''
        \begin{description}
            \item[τοποθετώ:] auf den (richtigen) Platz stellen
            \item[ταξινομώ:] die (richtige) Ordnung benennen
            \item[συμπεραίνω:] die (richtige) Schlussfolgerung ziehen
        \end{description}
    \end{exampleblock}
\end{frame}

\begin{frame}{Auch richtige Zahlen können sinnlos sein}
    \begin{exampleblock}{Frage: Wieviele Päpste je Quadratkilometer leben im Vatikan?}
        Die Antwort ($\frac{1 Papst}{0,44 km^{2}}$ = 2,27) ist ebenso richtig wie sinnlos.
    \end{exampleblock}
     \begin{exampleblock}{Frage: Wie drückt man einen Notendurchschnitt von 2,25 auf einer Skala von 1-5 aus?}
        $\frac{2,25}{5}*6$ = 1,875
    \end{exampleblock}
     \begin{exampleblock}{Frage: ... und wie auf einer Skala von A-E?}
        ???
    \end{exampleblock}
    \begin{alertblock}{Alter chinesischer Fluch:}
        Mögest du in statistisch interessanten Zeiten leben!
    \end{alertblock}
    
\end{frame}