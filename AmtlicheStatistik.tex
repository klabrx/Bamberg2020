\section{Ebenen der amtlichen Statistik}
    \begin{frame}{EU}
   \metroset{block=fill}
        \begin{block}{Eurostat}
            \begin{description}
                \item[Sitz:] Luxemburg
                \item[Rang:] Generaldirektion der Europäischen Kommission
                \item[Aufgabe:] Standardisierung, Verarbeitung und Veröffentlichung vergleichbarer Daten auf EU-Ebene
                \item[Arbeitsweise:] führt selbst \alert{keine} Erhebungen durch, macht Vorgaben an die Mitgliedsländer
            \end{description}
        \end{block}
   \end{frame}
    \begin{frame}{Bund}
    \metroset{block=fill}
        \begin{block}{Statistisches Bundesamt - destatis}
            \begin{description}
                \item[Sitz:] Wiesbaden
                \item[Rang:] Bundesamt, dem BMI unterstellt
                \item[Aufgabe:] siehe \S1 BStatG
                \item[veröffentlicht] verbindliche Klassifikationen, Verzeichnisse, Systematiken, z.B.
                    \begin{itemize}
                        \item Güterverzeichnis für Produktionsstatistiken (GP2019)
                        \item Klassifikation der Wirtschaftszweige (WZ2008)
                        \item Klassifikation der Berufe (KldB2010)
                        \item Staats- und Gebietssystematik
                        \item Gemeindeverzeichnis-Informationssystem GV-ISys (aka ``AGS'')
                    \end{itemize}
            \end{description}
        \end{block}
    \end{frame}
    \begin{frame}[allowframebreaks]{Länder}
    \metroset{block=fill}
        \begin{block}{Statistische Landesämter}
            \begin{itemize}
                \item Baden-Württemberg (Stuttgart)
                \item Bayern (Fürth)
                \item Berlin-Brandenburg (Potsdam)
                \item Bremen (Bremen)
                \item Hamburg und Schleswig-Holstein (``Statistik Nord'' in Hamburg und Kiel)
                \item Hessen (Wiesbaden)
                \item Mecklenburg-Vorpommern (Schwerin)
            \end{itemize}
        \end{block}            
    \framebreak
        \begin{block}{Statistische Landesämter (cont.)}
            \begin{itemize}            
                \item Niedersachsen: (Hannover)
                \item Nordrhein-Westfalen (``Landesbetrieb IT-NRW'' in Düsseldorf)
                \item Rheinland-Pfalz (Bad Ems)
                \item Saarland (Saarbrücken)
                \item Sachsen (Kamenz)
                \item Sachsen-Anhalt (Halle an der Saale)
                \item Thüringen (Erfurt)
            \end{itemize}
        \end{block}
    \framebreak
        \begin{block}{Profil der StatLÄ}
            \begin{description}
                \item[Rang:] Im Allgemeinen eine Oberste Landesbehörde
                \item[Aufgabe:] Im jeweiligen Landesstatikgesetz\footnote{z.B. \S2 Abs. 1 Satz 1 LStatG NRW: Die Landesstatistik hat die Aufgabe, Daten zu erheben, zu sammeln, aufzubereiten, darzustellen und zu analysieren, soweit Landesrecht dies bestimmt.} geregelt
                \item[Schnittstelle:] Da der Bund Aufgaben nicht direkt an die Gemeinden übertragen kann, agieren die StatLÄ als Mittler, z.B. beim Zensus
            \end{description}
        \end{block}
    \end{frame}
    \begin{frame}{Kreise und Kommunen}
    \metroset{block=fill}
        \begin{block}{Organisationsform}
            von der kleinen Einpersonenstelle bis hin zum großen, dem Direktorium unmittelbar unterstellten statistischen Amt ...
        \end{block}
        \begin{block}{Rang, Hierarchie}
            Stabstelle, Fachabteilung, Amt, Referat, Fachgruppe, ... 
        \end{block}        
        \begin{block}{Aufgabe}
            in den jeweiligen Satzungen und Dienstanweisungen geregelt
        \end{block}    
        \begin{block}{Alleinstellungsmerkmale}
            \begin{itemize}
                \item Auswertungen in kleinräumiger, untergemeindlicher Gliederung
                \item Detaillierte Ortskenntnis\footnote{Wo die Ortskenntnis der Landesämter mikroskopisch klein ist, ist die der Kommunalstatistik mikroskopisch genau.}
                \item Kurze Wege zu kommunalen Datenquellen
            \end{itemize}
           
            
        \end{block}    
  
    \end{frame}