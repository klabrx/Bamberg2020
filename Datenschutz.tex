\section{Datenschutz}
\begin{frame}{EU-DSGVO}
Die EU-DSGVO nennt den Begriff der Statistik immerhin neunmal im eigentlichen Gesetztestext (ohne Begründung und Fußnoten)
\begin{description}
    \item [Art. 5 Abs. 1 lit. b:] Statistik verstößt nicht grundsätzlich gegen Zweckbindung
    \item [Art. 5 Abs. 1 lit. e:] Statistik darf personenbezogene Daten länger speichern
    \item [Art. 9 Abs. 1  i.V.m. Abs. 2 lit. j:] Statistik darf unter Umständen auch besondere personenbezogene Daten verarbeiten 
    \item [Art. 14 Abs. 5:]  Statistik darf unter Umständen Auskünfte verweigern
    \item [Art. 17] Statistik muss unter Umständen nicht löschen und vergessen
    \item [Art. 21 Abs. 6] Statistik kann sich im öffentichen Interesse einem widerspruch widersetzen
  \end{description}
\end{frame}

\begin{frame}{Art. 89 EU-DSGVO}
    Garantien und Ausnahmen in Bezug auf die Verarbeitung zu im öffentlichen Interesse liegenden Archivzwecken, zu wissenschaftlichen oder historischen Forschungszwecken und zu \alert{statistischen} Zwecken
    \begin{enumerate}
        \item Die Verarbeitung \dots zu \alert{statistischen} Zwecken unterliegt geeigneten Garantien für die Rechte und Freiheiten der betroffenen Person gemäß dieser Verordnung. Mit diesen Garantien wird sichergestellt, dass technische und organisatorische Maßnahmen bestehen, mit denen insbesondere die Achtung des Grundsatzes der Datenminimierung gewährleistet wird. \dots
        \item Werden personenbezogene Daten \dots zu \alert{statistischen} Zwecken verarbeitet, können vorbehaltlich der Bedingungen und Garantien gemäß Absatz 1 \dots Ausnahmen von den Rechten gemäß der Artikel 15, 16, 18 und 21 vorgesehen werden\footnote{Auskunftsrecht, Recht auf Berichtigung, Recht auf Einschränkung der Bearbeitung, Widerspruchsrecht}, als diese Rechte voraussichtlich die Verwirklichung der spezifischen Zwecke unmöglich machen oder ernsthaft beeinträchtigen und solche Ausnahmen für die Erfüllung dieser Zwecke notwendig sind.
    \end{enumerate}
\end{frame}

